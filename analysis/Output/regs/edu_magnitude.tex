
\begin{table}[!htbp] \centering 
  \caption{Magnitude of the Hispanic achievement gap and SLTV effect size} 
  \label{t:edu_magnitude} 
	\scalebox{.8}{
		\begin{threeparttable}
			\begin{tabular}{lcc|cccccccc}
				\hline\hline\addlinespace
				& Asian-Hispanic gap & Gap after SLTV & White-Hispanic gap\\
				\cmidrule(lr){2-2}\cmidrule(lr){3-3}\cmidrule(lr){4-4}&  (1) & (2) & (3)  \\
				\addlinespace\hline\addlinespace
				SAT/ACTs taken & 46.8\% & 38.3\% & 36.6 \% \\
				Calculus taken & 53.6\% & 41.0\% & 15.0\% \\
				APs passed & 72.3\% & 69.6\% & 17.8\% \\
				Gifted students & 60.5\% & 51.0\% & 56.6\% \\
				Advanced math taken & 45.3\% & 31.7\% & 25.8\% \\
				Biology taken & 5.6\% & -18.9\% & -6.2\% \\
				Physics taken & 43.7\% & 26.2\% & 25.4\% \\
				Chemistry taken & 27.7\% & 6.7\% & 9.9\% \\
					\addlinespace\hline\hline
			\end{tabular}
			\begin{tablenotes}[flushleft]
				\item \textit{Notes:} The table presents the achievement gap in percentage terms between Asians and Hispanics in column 1, and between whites and Hispanics in column 3. Column 2 presents the achievement gap using regression coefficients from Table~\ref{t:edu_main}, Column 1 for SAT/ACTs taken, calculus taken, and APs passed, and coefficients from Appendix Table~\ref{t:edu_extra_achieve} for the remaining outcomes.		
					\end{tablenotes}
		\end{threeparttable}
	}
\end{table}