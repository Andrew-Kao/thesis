
\begin{table}[!htbp] \centering 
  \caption{Magnitude of the Hispanic achievement gap and SLTV effect size} 
  \label{t:edu_magnitude} 
	\scalebox{.8}{
		\begin{threeparttable}
			\begin{tabular}{lc|ccccccccc}
				\hline\hline\addlinespace
				& White-Hispanic gap & Asian-Hispanic gap & Gap after SLTV \\
				\cmidrule(lr){2-2}\cmidrule(lr){3-3}\cmidrule(lr){4-4}&  (1) & (2) & (3)  \\
				\addlinespace\hline\addlinespace
				SAT/ACTs taken & 36.6 \% & 46.8\% & 38.3\%  \\
				Calculus taken & 15.0\% & 53.6\% & 41.0\%  \\
				APs passed & 17.8\% & 72.3\% & 69.6\%  \\
				Gifted students & 56.6\% & 60.5\% & 51.0\%  \\
				Advanced math taken & 25.8\% & 45.3\% & 31.7\%  \\
				Biology taken  & -6.2\% & 5.6\% & -18.9\% \\
				Physics taken & 25.4\% & 43.7\% & 26.2\%  \\
				Chemistry taken & 9.9\%  & 27.7\% & 6.7\% \\
					\addlinespace\hline\hline
			\end{tabular}
			\begin{tablenotes}[flushleft]
				\item \textit{Notes:} The table presents the achievement gap in percentage terms between whites and Hispanics in column 1, and between Asian and Hispanics in column 2. Column 3 presents the achievement gap using regression coefficients from Table~\ref{t:edu_main}, Column 1 for SAT/ACTs taken, calculus taken, and APs passed, and coefficients from Appendix Table~\ref{t:edu_extra_achieve} for the remaining outcomes.		
					\end{tablenotes}
		\end{threeparttable}
	}
\end{table}