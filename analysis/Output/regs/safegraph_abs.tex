
\begin{table}[!htbp] \centering 
  \caption{Effect of Spanish language TV on Hispanic foot traffic} 
  \label{t:safegraph_abs} 
	\scalebox{.75}{
		\begin{threeparttable}
			\begin{tabular}{lcccccccccc}
				\hline\hline\addlinespace
				& \multicolumn{4}{c}{\textit{IHS(visitors to location)}}  \\  				
				\cmidrule(lr){2-5} &  (1) & (2) & (3) & (4) \\
				\addlinespace\hline\addlinespace
				\multicolumn{4}{l}{Panel A.1: Restaurants --- Hispanic dummy} \\ 
                              	\hline\addlinespace
				TV $\times$ Hispanic food&     0.829***&       0.829***&       0.829***&       0.829***\\
                    &     (0.151)   &     (0.151)   &     (0.151)   &     (0.151)   \\
				\addlinespace\hline
				\multicolumn{4}{l}{Panel A.2: Restaurants --- Japanese dummy} \\ 
                              	\hline\addlinespace
				TV $\times$ Japanese food &       0.084   &       0.084   &       0.084   &       0.084   \\
                    &     (0.183)   &     (0.183)   &     (0.183)   &     (0.184)   \\
				\addlinespace\hline
				\multicolumn{4}{l}{Panel A.3: Restaurants --- Brazilian dummy} \\ 
                              	\hline\addlinespace
				Hispanic $\times$ TV $\times$ Brazilian food&   0.927*** & 0.927*** & 0.927*** & 0.927*** \\
				& (0.183) &  (0.183) &  (0.183) &  (0.184) \\
 				\addlinespace\hline
				\multicolumn{4}{l}{Panel A.4: Restaurants --- Cajun and Creole dummy} \\ 
                              	\hline\addlinespace
				TV $\times$ Cajun and Creole food&     -0.240 & -0.240 & -0.240 & -0.240 \\
				& (0.409) & (0.409) & (0.409) & (0.410) \\
				\addlinespace\hline
				N & 101,618 & 101,618 & 101,618 & 101,618 \\ 
				\hline\hline\addlinespace
				\multicolumn{4}{l}{Panel B.1: Recreation --- Hispanic dummy} \\ 
                              	\hline\addlinespace
				TV $\times$ Hispanic brand&       0.885***&       0.885***&       0.885***&       0.885***\\
                    &     (0.327)   &     (0.327)   &     (0.327)   &     (0.327)   \\
                    			\addlinespace\hline
				\multicolumn{4}{l}{Panel B.2: Recreation --- Japanese dummy} \\
                              	\hline\addlinespace
				TV $\times$ Japanese brand&       2.360***&       2.360***&       2.360***&       2.360***\\
                    &     (0.409)   &     (0.409)   &     (0.409)   &     (0.409)   \\
				\addlinespace\hline
				\multicolumn{4}{l}{Panel B.3: Recreation --- Brazilian dummy} \\
                              	\hline\addlinespace
				TV $\times$ Brazilian brand&        0.077   &       0.077   &       0.077   &       0.077   \\
                    &     (0.498)   &     (0.498)   &     (0.498)   &     (0.499)   \\
				\addlinespace\hline
				\multicolumn{4}{l}{Panel B.4: Recreation --- Cajun and Creole dummy} \\ 
                              	\hline\addlinespace
				TV $\times$ Cajun and Creole brand&      -0.550 & -0.550 & -0.550 & -0.550 \\
				& (2.051)  & (2.051)& (2.051) & (2.053) \\
				\addlinespace\hline
				N & 34,990 & 34,990 & 34,990 & 34,990 \\ 
				\hline\hline\addlinespace
				County log(income) & Yes & Yes & Yes & Yes \\
				County \% Hispanic & No & Yes & Yes & Yes \\
				County log(pop.) & No & No & Yes & Yes \\
				County FE & No & No & No & Yes \\
				NAICS code FE & No & No & No & Yes \\
					\addlinespace\hline\hline
			\end{tabular}
			\begin{tablenotes}[flushleft]
				\item \textit{Notes:} The table presents coefficient estimates from regressions at the establishment-visitor identity level, where a visitor identity is one of 2 categories (TV or not), only keeping locations within 100 KM of a Spanish language TV contour boundary. The dependent variable are inverse hyperbolic sine transformed counts of Hispanic visitors to a given location. Panel A restricts the universe of locations to food service establishments, while Panel B restricts to arts, entertainment, and recreation establishments. TV dummy is an indicator variable for visitors to the location with home access to Spanish language television. Panels A.1 and B.1 interacts this variable with an indicator for Hispanic establishments, Panels A.2 and B.2 interacts this variable with an indicator for Japanese establishments, Panels A.3 and B.3 interacts this variable with an indicator for Brazilian establishments, and Panels A.4 and B.4 interact this variable with an indicator for Cajun and Creole establishments.   Columns 1-4 include controls for the mean log(income) of the county. Columns 2-4 control for the percentage of the county that is Hispanic. Columns 3-4 control for the county's log(population). Column 4 adds fixed effects for the county and NAICS code. Standard errors are clustered at the county level. *, **, and *** denote statistical significance at the 10\%, 5\%, and 1\% levels, respectively.
			\end{tablenotes}
		\end{threeparttable}
	}
\end{table}