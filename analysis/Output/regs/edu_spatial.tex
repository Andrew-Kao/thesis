\begin{table}[!h]
	\centering
	\captionsetup{skip=1.5pt}
	\caption{Spatial Robustness of Influence of Spanish Language Television on Hispanic Victims of Ethnicity-Based Harassment} \label{edu_spatial}
	\scalebox{.8}{
		\begin{threeparttable}
			\begin{tabular}{lcccccccccc}
				\hline\hline\addlinespace
				 & \multicolumn{3}{c}{\textit{IHS(\# Hispanic Victims of Harassment)}}\\
				&  (1) & (2) & (3) \\
                                \hline\addlinespace
 TV Dummy & 0.003$^{**}$ & 0.002$^{***}$ & 0.003$^{*}$ \\ 
  & (0.001) & (0.001) & (0.002) \\ 
 TV Dummy $\times$ Distance to Boundary & $-$0.0001$^{**}$ & $-$0.0001$^{***}$ & $-$0.0001$^{**}$ \\ 
  & (0.00002) & (0.00001) & (0.00003) \\ 
\hline\hline\addlinespace
Observations & 40,811 & 40,811 & 40,811 \\ 
Log Likelihood &  & $-$4,304.916 & $-$4,299.820 \\ 
$\sigma^{2}$ &  & 0.072 & 0.072 \\ 
Akaike Inf. Crit. &  & 8,629.833 & 8,619.640 \\ 
Wald Test (df = 1) &  & 686.149$^{***}$ & 686.981$^{***}$ \\ 
LR Test (df = 1) &  & 657.312$^{***}$ & 667.505$^{***}$ \\ 
\hline \addlinespace
                                County/School Controls & Yes & Yes  & Yes \\
                                Model & OLS & SAR Lag & SAR Error \\
				\addlinespace\hline\hline
			\end{tabular}
			\begin{tablenotes}[flushleft]
				\item \textit{Notes:} The table presents coefficient estimates from regressions at the school level, only keeping schools within 100 KM of a contour boundary. The dependent variable is the inverse hyperbolic sine transformed counts of Hispanic students who are bullied or harassed on the basis of their ethnicity. The key dependent variable of interest is the TV Dummy, which tracks whether the school is within a coverage contour boundary for a Spanish language television station. This is interacted with the distance to the boundary. County and school controls include log income, log population, percentage county Hispanic for the county which the school is located in, and the number of Hispanic students in the school. Additionally controlling for number of teachers, total number of students at the school, and dummies for whether the school contains a primary, middle, and high school division yields similar coefficients, although standard errors cannot be estimated due to computational limitations. The SAR Lag model is a spatial autoregressive lag model and the SAR Error model is a spatial autoregressive error model, both with weight matrices based on 4 nearest neighbours. Standard errors are given in parentheses. *, **, and *** denote statistical significance at the 10\%, 5\%, and 1\% levels, respectively.
			\end{tablenotes}
		\end{threeparttable}
	}
\end{table}