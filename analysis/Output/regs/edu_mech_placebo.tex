
\begin{table}[!htbp] \centering 
  \caption{Effect of Spanish language TV on Hispanic vs. Asian, placebo identity outcomes} 
  \label{t:edu_mech_placebo} 
	\scalebox{.8}{
		\begin{threeparttable}
			\begin{tabular}{lcccccccccc}
				\hline\hline\addlinespace
%				& \multicolumn{4}{c}{\textit{Minutes of TV watched}}  \\  				\cmidrule(lr){2-5} 
				&  (1) & (2) & (3)  \\
				\addlinespace\hline\addlinespace
				\multicolumn{3}{l}{Panel A: IHS(IDEA (disability) students)} \\ %edu_dda_satactOLSIHS_spec3
                              	\hline\addlinespace
				TV dummy $\times$ Hispanic & 0.0318 & 0.0325 & 0.0318\\
  &(0.0338) & (0.0339) & (0.0338)\\
				\addlinespace\hline
				N & 81,622 & 81,622 & 81,622 \\ 
				\hline\hline\addlinespace
				\multicolumn{3}{l}{Panel B: IHS(bullied based on sex)} \\ %edu_dda_calcOLSIHS_spec3
                              	\hline\addlinespace
				TV dummy $\times$ Hispanic & 0.0090 & 0.0088 & 0.0088\\
  &(0.0056) & (0.0055) & (0.0055)\\	 
				   \addlinespace\hline
				N & 22,168 & 22,168 & 22,168 \\ 
				\hline\hline\addlinespace
				School district FE & Yes & Yes  & Yes\\
				\# Hispanic, Asian students & Yes & Yes  & Yes\\
                                	School size controls & No & Yes & Yes\\
                                	School type controls & No & No & Yes \\
					\addlinespace\hline\hline
			\end{tabular}
			\begin{tablenotes}[flushleft]
				\item \textit{Notes:} The table presents coefficient estimates from regressions at the school-ethnicity level, only keeping schools within 100 KM of a Spanish language TV contour boundary. The dependent variable are inverse hyperbolic sine transformed counts of students classified under IDEA (students with disabilities) in Panel A and the number of students bullied on the basis of their sex in Panel B. TV dummy is an indicator variable for a school with access to Spanish language television, which is interacted with an indicator for whether the demographic is Hispanic (the omitted group are Asians). Columns 1-3 control for the number of Hispanic and Asian students enrolled. Columns 2-3 control for the number of teachers and total number of students at the school. Column 3 controls for indicators denoting whether the school contains a primary, middle, and high school division. School district fixed effects are always included. Standard errors are clustered at the school district level. *, **, and *** denote statistical significance at the 10\%, 5\%, and 1\% levels, respectively.
			\end{tablenotes}
		\end{threeparttable}
	}
\end{table}