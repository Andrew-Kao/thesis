
\begin{table}[!htbp] \centering 
  \caption{Effect of Spanish language TV on Hispanic vs. Asian academic achievement} 
  \label{t:edu_extra_achieve} 
	\scalebox{.75}{
		\begin{threeparttable}
			\begin{tabular}{lcccccccccc}
				\hline\hline\addlinespace
%				& \multicolumn{4}{c}{\textit{Minutes of TV watched}}  \\  				\cmidrule(lr){2-5} 
				&  (1) & (2) & (3)  \\
				\addlinespace\hline\addlinespace
				\multicolumn{3}{l}{Panel A: IHS(gifted students)} \\ %edu_dda_giftedOLSIHS_spec3
                              	\hline\addlinespace
				 TV dummy $\times$ Hispanic & 0.2389$^{***}$ & 0.2389$^{***}$ & 0.2389$^{***}$\\
  &(0.0284) & (0.0284) & (0.0284)\\
				  \addlinespace\hline
				N & 52,130 & 52,130 & 52,130 \\ 
				\hline\hline\addlinespace
				\multicolumn{3}{l}{Panel B: IHS(advanced math courses)} \\ %edu_dda_satactOLSIHS_spec3
                              	\hline\addlinespace
				 TV dummy $\times$ Hispanic & 0.2501$^{***}$ & 0.2501$^{***}$ & 0.2501$^{***}$\\
  &(0.0362) & (0.0362) & (0.0362)\\
				\addlinespace\hline
				N & 14,354 & 14,354 & 14,354 \\ 
				\hline\hline\addlinespace
				\multicolumn{3}{l}{Panel C: IHS(biology courses)} \\ %edu_dda_appOLSIHS_spec3
                              	\hline\addlinespace
				 TV dummy $\times$ Hispanic & 0.2596$^{***}$ & 0.2596$^{***}$ & 0.2596$^{***}$\\
  &(0.0272) & (0.0272) & (0.0272)\\
				\addlinespace\hline
				N & 19,008 & 19,008 & 19,008 \\ 
				\hline\hline\addlinespace
				\multicolumn{3}{l}{Panel D: IHS(physics courses)} \\ %edu_dda_appOLSIHS_spec3
                              	\hline\addlinespace
				 TV dummy $\times$ Hispanic & 0.3114$^{***}$ & 0.3114$^{***}$ & 0.3114$^{***}$\\
  &(0.0345) & (0.0345) & (0.0345)\\
				\addlinespace\hline
				N & 13,952 & 13,952 & 13,952 \\ 
				\hline\hline\addlinespace
				\multicolumn{3}{l}{Panel E: IHS(chemistry courses)} \\ %edu_dda_appOLSIHS_spec3
                              	\hline\addlinespace
				 TV dummy $\times$ Hispanic & 0.2896$^{***}$ & 0.2896$^{***}$ & 0.2896$^{***}$\\
  &(0.0273) & (0.0273) & (0.0273)\\
				\addlinespace\hline
				N & 16,472 & 16,472 & 16,472 \\ 
				\hline\hline\addlinespace
				 School district FE & Yes & Yes  & Yes\\
				\# Hispanic, Asian students & Yes & Yes  & Yes\\
                                	School size controls & No & Yes & Yes\\
                                	School type controls & No & No & Yes \\
					\addlinespace\hline\hline
			\end{tabular}
			\begin{tablenotes}[flushleft]
				\item \textit{Notes:} The table presents coefficient estimates from regressions at the school-ethnicity level, only keeping schools within 100 KM of a Spanish language TV contour boundary. The dependent variable are inverse hyperbolic sine transformed counts of the number of gifted students in Panel A, the number of students enrolled in an advanced math course in Panel B, the number of students enrolled in a biology course in Panel C, the number of students enrolled in a physics course in Panel D, and the number of students enrolled in a chemistry course in Panel E. TTV dummy is an indicator variable for a school with access to Spanish language television, which is interacted with an indicator for whether the demographic is Hispanic (the omitted group are Asians). Columns 1-3 control for the number of Hispanic and Asian students enrolled. Columns 2-3 control for the number of teachers and total number of students at the school. Column 3 controls for indicators denoting whether the school contains a primary, middle, and high school division. School district fixed effects are always included. Standard errors are clustered at the state level. *, **, and *** denote statistical significance at the 10\%, 5\%, and 1\% levels, respectively.		\end{tablenotes}
		\end{threeparttable}
	}
\end{table}