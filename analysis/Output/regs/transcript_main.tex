
\begin{table}[!htbp] \centering 
  \caption{Differential effect of Spanish language TV by program content on Hispanic vs. Asian academic achievement} 
  \label{t:transcript_main} 
	\scalebox{.8}{
		\begin{threeparttable}
			\begin{tabular}{lcccccccccc}
				\hline\hline\addlinespace
%				& \multicolumn{4}{c}{\textit{Minutes of TV watched}}  \\  				\cmidrule(lr){2-5} 
				&  (1) & (2) & (3)  \\
				\addlinespace\hline\addlinespace
				\multicolumn{3}{l}{Panel A: IHS(SAT/ACTs taken)} \\ %edu_dda_satactOLSIHS_spec3
                              	\hline\addlinespace
				 TV $\times$ Hispanic $\times$ \% programs on identity & 2.313$^{**}$ &  &  \\ 
				  & (0.943) &  &  \\ 
				 TV $\times$ Hispanic $\times$ \% programs on education &  & $-$0.516 &  \\ 
				  &  & (0.626) &  \\ 
				 TV $\times$ Hispanic $\times$ \% programs with role models &  &  & $-$2.085 \\ 
				  &  &  & (2.151) \\ 
				\addlinespace\hline
				N & 21,610 & 21,610 & 21,610 \\ 
				\hline\hline\addlinespace
				\multicolumn{3}{l}{Panel B: IHS(calculus taken)} \\ %edu_dda_appOLSIHS_spec3
                              	\hline\addlinespace
				 TV $\times$ Hispanic $\times$ \% programs on identity & 2.788$^{***}$ &  &  \\ 
				  & (1.034) &  &  \\ 
				 TV $\times$ Hispanic $\times$ \% programs on education &  & 0.829 &  \\ 
				  &  & (0.666) &  \\ 
				 TV $\times$ Hispanic $\times$ \% programs with role models &  &  & 1.616 \\ 
				  &  &  & (2.463) \\ 
				\addlinespace\hline
				N & 7,112 & 7,112 & 7,112 \\ 
				\hline\hline\addlinespace
				\multicolumn{3}{l}{Panel C: IHS(APs passed)} \\ %edu_dda_calcOLSIHS_spec3
                              	\hline\addlinespace
				 TV $\times$ Hispanic $\times$ \% programs on identity & 1.721 &  &  \\ 
				  & (1.280) &  &  \\
				 TV $\times$ Hispanic $\times$ \% programs on education &  & 0.903 &  \\ 
				  &  & (0.922) &  \\ 
				 TV $\times$ Hispanic $\times$ \% programs with role models &  &  & $-$1.184 \\ 
				  &  &  & (2.989) \\ 
				  \addlinespace\hline
				N & 3,168 & 3,168 & 3,168 \\ 
				\hline\hline\addlinespace
				School district FE & Yes & Yes  & Yes\\
				\# Hispanic, Asian students & Yes & Yes  & Yes\\
                                	School size controls & No & Yes & Yes\\
                                	School type controls & No & No & Yes \\
					\addlinespace\hline\hline
			\end{tabular}
			\begin{tablenotes}[flushleft]
				\item \textit{Notes:} The table presents coefficient estimates from regressions at the school-ethnicity level, only keeping schools within 100 KM of a Spanish language TV contour boundary. The dependent variable are inverse hyperbolic sine transformed counts of the number of students taking the SAT or ACT in Panel A, the number of students enrolled in calculus in Panel B, and the number of Advanced Placement tests passed in Panel C. \% programs on identity, education, and role models are coded based on TV channel network transcripts. TV dummy is an indicator variable for a school with access to Spanish language television, which is interacted with an indicator for whether the demographic is Hispanic (the omitted group are Asians) and the \% of programs on identity, education, and role models. Columns 1-3 control for the number of Hispanic and Asian students enrolled. Columns 2-3 control for the number of teachers and total number of students at the school. Column 3 controls for indicators denoting whether the school contains a primary, middle, and high school division. School district fixed effects are always included. Standard errors are clustered at the school district level. *, **, and *** denote statistical significance at the 10\%, 5\%, and 1\% levels, respectively.			\end{tablenotes}
		\end{threeparttable}
	}
\end{table}