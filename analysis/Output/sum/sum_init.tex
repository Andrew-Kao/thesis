\begin{table}[!h]
	\centering
	\captionsetup{skip=1.5pt}
	\caption{Summary Statistics}\label{sum_init}
	\scalebox{.8}{
		\begin{threeparttable}
			\begin{tabular}{l@{\extracolsep{4pt}}cccc}
				\hline\hline\addlinespace
				& \textit{All} &  \textit{No TV} & \textit{TV} \\
				\cline{2-4} \addlinespace
				& (1) & (2) & (3) \\
				Panel A: Migrations & \multicolumn{3}{c}{} \\
				\hline\addlinespace
				IHS(Hispanic Migrants) & 4.331 & 4.035 & 4.462 \\
				& 1.297 & 1.204 & 1.316 \\
				Log Income & 9.513 & 9.449 & 9.541 \\
				& (0.281) & (0.201) & (0.305) \\
				Log Population & 12.358 & 11.942 & 12.542 \\
				& (1.516) & (1.640) & (1.419) \\
				Fraction County Hispanic & 0.124 & 0.085 & 0.141 \\
				& (0.155) & (0.122) & (0.164) \\
				\hline\addlinespace
				 \hline\addlinespace
			\end{tabular}
			\begin{tablenotes}[flushleft]
				\item \textit{Notes:} The table presents means (and standard deviations). Variables in Panel A  are data from counties within 100 KM of a coverage contour. Columns 2 and 3 show data for the subsample without and with SLTV coverage, respectively. No control is significantly different across the coverage contour at the $\alpha = .1$ level.
			\end{tablenotes}
		\end{threeparttable}
	}
\end{table}
