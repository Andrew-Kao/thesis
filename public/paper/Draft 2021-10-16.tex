

\documentclass[11pt]{article}

\ExplSyntaxOn % providing \expandableinput
\cs_new:Npn \expandableinput #1
  { \use:c { @@input } { \file_full_name:n {#1} } }
\ExplSyntaxOff

\usepackage{amsmath,amsthm,amssymb, pdfpages,mathtools}
\usepackage{color}
\usepackage{array}
\usepackage{gastex}
\usepackage{subfigure}
\usepackage[normalem]{ulem}
\usepackage{xcolor,psfrag,graphicx}
\usepackage{setspace}
\usepackage{natbib}
\usepackage{lscape}
\usepackage{enumerate}
\usepackage{appendix}
\usepackage[hidelinks, hypertexnames=false]{hyperref}
\usepackage{lscape}
\usepackage{tabularx}
\usepackage{threeparttable}
\usepackage{caption}
\usepackage{booktabs}

\setcounter{MaxMatrixCols}{10}

\setlength{\evensidemargin}{0.0in}
 \setlength{\oddsidemargin}{0.0in}
 \setlength{\textwidth}{6.5in}
 \topmargin -0.25in
 \textheight 8.5in
 \hfuzz=50pt
 \pagestyle{plain}
\newcommand{\eqthreshn}{{t^*_N}}
\newcommand{\pnd}{1-p+pF(\eqthreshn)}
\newcommand{\ppnd}{\big(1-p+pF(\eqthreshn)\big)}
\newcommand{\eqmfreq}{{\omega^*}}
\newcommand{\eqmfreqn}{{\omega^*_N}}
\newcommand{\eqmfreqnp}{{\omega^*_{N+1}}}
\newcommand{\eqthresh}{{t^*}}
\newcommand{\eqthreshX}{{t^{**}}}
\newcommand{\nbar}{{\overline{N}}}
\newcommand{\wlim}{\omega_\infty}
\newcommand{\wdye}{\hat{\omega}}
\newcommand{\tdye}{\hat{t}}
\newcommand{\limn}{\lim_{N\to\infty}}
\newcommand{\fsn}{\omega_N^*}
\newcommand{\eqprize}{\phi^*}
\newcommand{\moprize}{\phi^M}
\newcommand{\dif}{\;\mathrm{d}}
\newcommand{\diffp}[2]{\frac{\partial #1}{\partial #2}}
\newcommand{\diff}[2]{\frac{\dif #1}{\dif #2}}
\renewcommand{\Re}{\mathbb{R}}                             
\def\endproof{{\quad}$\blacksquare$}
\newcommand{\indicator}[1]{\mathbbm{1}_{\left[ {#1} \right]}}
\newtheorem{theorem}{Theorem}
\newtheorem{proposition}{Proposition}
\newtheorem{prop}{Proposition}
\newtheorem{example}{Example}
\newtheorem{assumption}{Assumption}
\newtheorem{corollary}[theorem]{Corollary}
\newtheorem{acknowledgement}[theorem]{Acknowledgement}
\newtheorem{definition}{Definition}
\newtheorem{lemma}{Lemma}
\newtheorem{remark}{Remark}
\newtheorem{condition}[theorem]{Condition}
 \setlength{\evensidemargin}{0.0in}
 \setlength{\oddsidemargin}{0.0in}
 \setlength{\textwidth}{6.5in}
 \topmargin -0.25in
 \textheight 8.5in
 \hfuzz=50pt
 \pagestyle{plain}
\newcommand{\Change}[1]{{\color{red}#1}}
\renewcommand{\theenumi}{\roman{enumi}}            
\renewcommand{\labelenumi}{(\theenumi)}

%%%%%%%%%%%%%%%
% Page Format %
%%%%%%%%%%%%%%%
 \setlength{\evensidemargin}{0.0in}
 \setlength{\oddsidemargin}{0.0in}
 \setlength{\textwidth}{6.5in}
 \topmargin -0.25in
 \textheight 8.5in
 \hfuzz=50pt
 \pagestyle{plain}





\begin{document}
\title{\textbf{Seeing is Believing: Identity, Inequality, and the Impact of Television on the Hispanic Achievement Gap}%Unravel}
\thanks{Many appreciated suggestions, critiques and encouragement were provided by Leonardo Bursztyn, Lucas Cusimano, Benjamin Enke, Victor Lima, Aakaash Rao, Jaya Wen, David Yang, and Kotaro Yoshida, seminar participants at EPoD and Harvard PE, as well as Trip from SatelliteGuys for technical advice. }\\
}



\author{Andrew Kao\thanks{Harvard University. Email: \texttt{andrewkao@fas.harvard.edu}} }

%\begin{center}
\date{October 2021}
{\vspace{-5ex}}
%\end{center}


\maketitle

\begin{abstract}
\noindent  Can identity reduce inequality? Using a spatial regression discontinuity arising from a FCC regulation, I investigate the impact of Spanish Language Television (SLTV) on Hispanic students in public schools. I find that SLTV improves academic performance and helps close the Hispanic achievement gap. I marshal three of sources of evidence that each indicate an identity mechanism is at play: (1) more Hispanic students are labelled `limited English proficiency' and bullied on the basis of their ethnicity in SLTV schools, (2) Hispanic students perform better academically in locales where SLTV programming focuses more on the Hispanic identity, and (3) Hispanics with access to SLTV differentially visit Hispanic branded establishments more frequently. Collectively, they suggest that identity is a mechanism through which SLTV reduces the Hispanic achievement gap.\\\\
\textbf{JEL Codes:} I24, J15, L82, Z13.\\
\textbf{Keywords:} Hispanic, television, education, identity
\end{abstract}




\newsavebox{\tablebox} \newlength{\tableboxwidth}

\setlength{\baselineskip}{22pt}

\renewcommand{\thefootnote}{\fnsymbol{footnote}}


\thispagestyle{empty}

\newpage 
\renewcommand{\thefootnote}{\arabic{footnote}}

\pagebreak 
\setcounter{page}{0}


\onehalfspacing

%\tableofcontents

\newpage

\setcounter{page}{1}
\section{Introduction}

%\begin{quotation}
%\textit{[Television] has altered every phase of the American vision and identity. }
%\begin{flushright} - Marshall McLuhan, \textit{War and Peace in the Global Village}\end{flushright}
%\end{quotation}

The Hispanic achievement gap is wide and persistent. Hispanics face the lowest high school and college completion rates out of all major ethnic and racial groups in the United States.\footnote{See \cite{tienda_hispanicity_2009}. This achievement gap extends to many more educational outcomes, from test scores to disciplinary incidents to enrolment in advanced classes. Factors such as segregation \citep{cascio_cracks_2012}, XXX, XXX exacerbate the achievement gap, whereas interventions such as providing free computers \citep{fairlie2012academic} XXX help close it.  TODO: other cites on the achievement gap. } In this paper, I argue that Spanish Language Television (SLTV) is a media technology that has reduced educational inequality, and that moreover, these gains in Hispanic student performance can be attributed (at least in part) to a heightened sense of a Hispanic identity.
% TODO: use citations and figures in https://www.colorincolorado.org/article/closing-achievement-gap-focus-latino-students-0
% move footnote into main draft?

Despite the rise of the internet, broadcast Spanish Language TV remains an important fixture in Hispanic households. 78\% of Spanish-dominant households watch SLTV, while 50\% of multi-language Spanish-speaking homes do. In 2010, every single one of the top 10 shows watched by Hispanics were Spanish language programs \citep{pardo_three_2011}. By investigating Spanish Language TV, I take a closer look at Hispanic communities and examine the ties between SLTV, identity, and education. 

To identify the causal effect of SLTV, I follow \cite{velez_tuning_2019} and exploit a spatial regression discontinuity arising from the FCC regulation OET Bulletin 69. This regulation grants federal protection of a TV station’s broadcast signal only within a certain distance of a station’s main antenna. Thus, households and schools just inside a TV station's coverage contour should be observably similar to those just outside the contour, except for the presence of broadcast and satellite TV. I argue that this allows me to identify the causal effect of SLTV, given several features: (1) contours are mechanically decided by a formula involving geographical features and antenna strength, (2) contours are large and tend to cut across suburban and `small town' areas, rather than dense urban areas which corporations might try to include for profitability reasons (urban centers fall squarely within contours), (3) SLTV stations were often built before this regulation was imposed, (4) demographic and other controls across the regression discontinuity are similar, and (5) Hispanics do not differentially migrate across contours, minimizing the possibility of selection. To further dispel concerns over potential confounds, I employ a difference-in-discontinuities design, comparing outcomes for Hispanic students against Asian students in schools within 100 kilometers of SLTV coverage contours.\footnote{I compare against Asian rather than white students because they are much less likely to identify as Hispanic.} 

I verify the relevance of this instrument's first stage by employing the difference-in-discontinuities approach with the American Time Use Dataset. I find that Hispanics watch over 10 minutes more of TV when within coverage contours, a plausible lower bound for the amount of extra Spanish Language TV watched if Hispanics substitute watching English programs with Spanish ones. I further show that Hispanics watch more TV with their children---Hispanic students, in other words. Notably, non-Hispanics do not exhibit differential TV viewership across SLTV coverage contours.

Next, I utilize the Civil Rights Data Collection to analyze the effect of SLTV on Hispanic students in public schools. Academic outcomes improve across the board: compared to Asians, Hispanics with SLTV are 16\% more likely to take the SAT or ACT, 27\% more likely to enrol in calculus and higher math, and pass 8\% more AP exams, among other results. These gains are present in absolute terms, extend to other academic outcomes, and also remain qualitatively similar under a variety of robustness tests, establishing that SLTV reduces the Hispanic achievement gap.

Given this finding, I then marshal three sources of evidence that each indicate an identity mechanism may drive these results. First, using the same data, I show that Hispanic students are differentially classified as having ‘limited English proficiency’ under SLTV despite greater general academic achievement, a likely outcome if these students shift from English to Spanish mastery due to the presence of SLTV. Furthermore, Hispanic students are also differentially bullied more on the basis of their ethnicity, directly indicating a more salient identity that other students may target. Second, I use archive.org’s TV transcript database to classify the proportion of programs in each SLTV station that focus on the Hispanic identity. I show that locales with a greater amount of SLTV programming focused on the Hispanic identity are associated with stronger Hispanic academic performance when compared against locales with fewer such programs on identity. However, a greater amount of programming focused on education \& schools or positive role models for children each have a null effect on Hispanic performance. This indicates that the content of these television programs matter, and that identity is a primary channel through which these gains are attained. Finally, I use foot-traffic data from Safegraph to demonstrate the general strengthening of the Hispanic identity under SLTV. Hispanics with SLTV are differentially more likely to visit Hispanic branded restaurants and recreation establishments. However, Hispanics with SLTV are no more likely to visit Brazilian, Greek, and Korean establishments. This speaks to the specific strengthening of the Hispanic identity versus a broader Latin American one. Collectively, these results suggest that identity is an important mechanism through which SLTV reduces inequality and the Hispanic achievement gap. 

\paragraph{Literature} Americans spend an average of three hours a day watching TV---more than any activity bar sleeping. Accordingly, a large literature has examined the impact that television has on education. Prior work has frequently been correlational and findings remain conflicted: one line of research contends that TV is as a distraction which `rots' the mind and harms student outcomes \citep{zavodny_does_2006},\footnote{ See \cite{aksoy2000panel}, \cite{hornik1981out}, and \cite{keith1986parental}. This theory enjoys popular support (see \cite{winn_plug-drug_2002} or \cite{gentile_well-child_2004} which finds broad support among paediatricians). \cite{huang2010dynamic} and \cite{nakamuro2015television} use causal identification strategies and also find negative (but generally much smaller) effects.} while another line of inquiry has found consistent null effects.\footnote{ \cite{gaddy1986television}, \cite{gortmaker1990impact}, and \cite{hu2020relationship} take correlational approaches, while \cite{munasib2010idiot} and \cite{kureishi2013does} use self-reportedly weak instruments.} \cite{gentzkow_preschool_2008} are closest to this paper in finding that TV improves student test scores---particularly among nonwhite students and English language learners. 

Others have studied the effect of television on Hispanic communities. \cite{oberholzer-gee_media_2009} demonstrate that the presence of Spanish language local news increases Hispanic voter turnout, whereas  \cite{velez_tuning_2019} (who develop the instrument used in this paper) find that SLTV depresses voter turnout. \cite{trujillo_devil_2012} run an experiment testing trust in the government and the census based on a scripted soap opera scene. I extend on this literature by moving beyond the political realm, arguing that the consequences of SLTV are large in educational settings too, and also provide the first evidence on a mechanism through which SLTV operates: identity.

There is a growing literature that looks at how identity can influence behaviour: this has been studied through theory, in the lab, and the field.\footnote{ See \cite{akerlof2000economics}, \cite{benjamin_social_2007}, \cite{benjamin_religious_2010}, and \cite{bursztyn2019moral}.} However, the underlying factors that build identity (rather than simply triggering them via priming or other short-term interventions) are less well understood. \cite{bisin_bend_2010}, \cite{atkin_how_2019}, and \cite{bazzi_unity_2019} encompass some recent studies on this topic, and all come to the conclusion that intergroup tensions or differences lead to a strengthening of identity. \cite{alesina2013origins} take the long view and shows how gender norms can be traced back to early agricultural practices. I contribute to this literature by proposing an alternate, media-based channel through which identity may be strengthened and influence action. This is closest to work such as \cite{jensen_power_2009} and \cite{gentzkow_media_2004}, which establish a link between media and gender norms, and media and anti-Americanism, respectively.\footnote{Other related work on the impact of mass media on social forces include \cite{ferrara_soap_2012}, \cite{kearney_media_2015},  \cite{olken_television_2009}, \cite{dellavigna_fox_2007},  \cite{yanagizawa-drott_propaganda_2014}, and \cite{putnam_bowling_2001}. For an overview, see \cite{dellavigna_economic_2015}.} Finally, in the education and psychology literature, the stereotype threat phenomenon tightly links minority identity and achievement gaps together (\cite{spencer2016stereotype}, \cite{appel2012stereotypes}), leading to educators to suggest methods such as ``situational disengagement'' to improve performance \citep{nussbaum2007situational}. This paper suggests that a stronger sense of identity may not have uniformly negative consequences on Hispanic students, creating space for a more positive conception of identity.


% some pushback claiming that individual shows can lead to fewer teenage births \citep{kearney_media_2015}




% TODO: caveats, include in relevant section
% - other outcomes of importance, violence and obesity.  --
% - Prove the existence of identity mechanism, but does not rule out other mechanisms






 

% Market sizing: CITE FCC Hispanic TV 
% The Hispanic community is made up of 13.96 million television households nationally, which account for approximately 12.2 percent of the 114.65 million television households in the United States (as of 2012)
% Nationwide, approximately 9.6 percent of U.S. television households are ?broadcast-only;? thus the overwhelming majority subscribe to a pay television service.14 The comparable figure nationwide for Hispanic households is 15.7 percent.15 
% roughly half do not use cable (page 29) - Alternative Distribution Service (ADS)/satellite and broadcast). We find that the majority of Hispanic television households appear to be either cable or Alternative Delivery System (ADS) households. ADS designations refer to households with one or more televisionsetsthatreceiveprogrammingfromoneoffourtypesofsystems: DBS;satellitedish (C-Band); satellite antenna television (SMATV); and multi-channel, multi-point distribution systems (MMDS). Broadcast television households,
% programming split fairly evenly between locally produced segments, news, telenovelas, and paid programming, the latter three of which may come from abroad

% viewed by half of all Spanish-dominant Latinos (http://www.horowitzresearch.com/press/spanish-language-tv-content-remains-integral-to-u-s-hispanics-tv-diet-new-horowitz-survey-shows/)
% better: Nielsen, 78% Spanish-dominant watch Spanish TV, 50% in multi-language homes, over 85% broadcast -- in 2010, top 10 broadcast shows in Hispanic demographic were all Spanish language
% market size: millions in LA, New York etc. https://www.statista.com/statistics/189824/largest-hispanic-television-markets-in-the-united-states-2011/

% broadcast/satellite TV vs cable: https://www.quora.com/What-is-the-difference-between-cable-and-broadcast

% in recent years, highest viewed: Pequeños Gigantes (talent show, kids), El Señor de los Cielos (telenovella, cartel leader),   https://www.statista.com/statistics/497739/spanish-tv-programs-usa/

% TODO: absolute gains in tables
% TODO: check caribbean?




%% Do everything with d^2
%% spatial errors: http://www.trfetzer.com/using-r-to-estimate-spatial-hac-errors-per-conley/


\paragraph{Layout.} Following this Introduction, Section \ref{s:data} presents the data sources used. Section \ref{s:rd} describes the difference-in-discontinuities empirical strategy and establishes the first stage. Section \ref{s:school} presents evidence that SLTV closes the Hispanic achievement gap. Section \ref{s:mech} presents evidence that an identity mechanism underlies these results. Finally, Section \ref{s:conclusion} concludes with a discussion of research questions opened by the preceding analysis.


\vspace{5em}
\hrule
\vspace{.5em}
EDIT \\
\hrule
\vspace{.5em}

\section{Data}\label{s:data}


\subsection{Broadcast TV and Geography}

The central instrument in this paper is the discontinuity in coverage contours of SLTV stations introduced via FCC regulation.

\paragraph{Coverage Contours} 

To build the coverage contours of SLTV stations (and thus find the boundaries across which people just receive/do not receive SLTV) in the US, we collected a list of the callsigns for all SLTV stations via the TMS API (TMS is a large provider of data on TV, movies, and other media).\footnote{ A TV station is defined to be SLTV if at least one of the primary broadcasts languages are Spanish.} There are 100 of these stations located across the United States. These callsigns were then matched against data from the FCC's OET Bulletin 69 and the FCC's CDBS Database to directly obtain the relevant coverage contour boundaries as prescribed and regulated by the FCC.\footnote{ 2015 coverage contour data is used due to the 'FCC Spectrum Repack' that began in 2018, which relocates a number of signals, affecting the reception and coverage for a substantial number of stations \citep{fletcher_fcc_2018}.} 
A map of all these contours can be seen in Figure \ref{contourfig}.

\paragraph{Geocoding}

Location data for all outcomes was collected in the form of addresses written in text. To transform this into proper spatial data (coordinates with latitude \& longitude), two geocoding tools were used: (1) ArcGIS, which has its own proprietary database of locations. Over 99\% of addresses were successfully matched to one location and geocoded. This was used to geocode the schooling data, as well as portions of the campaign contribution data. (2) The US Census Geocoder, which contains the census database of locations. Over 80\% of addresses were successfully matched to one location and geocoded.\footnote{The US Census geocoder, unlike the ArcGIS geocoder, is free. However, due to the higher precision of the ArcGIS geocoder, data constructed from it is used wherever possible. } This was used to geocode the business data, as well as portions of the campaign contribution data. It is unlikely for non-geocoded addresses to be correlated with the instrument, given the relatively narrow band around the contour retained for the spatial regression discontinuity.

For data that take the form of spatial points (such as the location of a school), determining its distance to the boundary and whether the datapoint falls within the coverage boundary is a straightforward process.  For data aggregated into grid points (typically a grid composed of $2 \times 2$ KM$^2$ chunks), we treat each any degree of intersection as the grid point falling within coverage, given its relatively small size. In locations covered by multiple SLTV stations, the distance to the boundary is taken as the distance to the closest boundary. %% FIX THIS: not true for migration. True for Trump, but smaller grid
% For data that cover a wider area (such as a county), in the standard specification, the area is said to fall within the coverage boundary if at least 50\% of its area does, and the distance from the area to the boundary is taken as the minimum distance from the boundary to the area.
% TODO: robustness; experiment with this for county controls data! calculate area... maybe also for campaigns


\subsection{Data}

The data on public schools comes from the US Department of Education's CRDC (Civil Rights Data Collection) dataset in 2015. In order to prevent discrimination and for transparency purposes, all public schools in the United States are required to report a substantial amount of information to the CRDC on an annualized basis.\footnote{ In practice, this data is not released to the public every year. Furthermore, not all schools report all data (or correct data) required of them, which is why the number of observations for different variables in this dataset fluctuates. Some data, such as that on AP examinations, are not mandatory, but the bulk of outcome variables are, with non-compliance on the mandatory data typically representing $<1\%$ of total data.} 

The dataset contains information on a total of 96,350 schools across 17,280 school districts. Figure \ref{schooldistrictfig} contains a map of these schools, while summary statistics for the outcomes and controls are presented in Panel B of Table \ref{sum_out}.

School level controls include the number of teachers, the number of total students, the number of Hispanic students, as well as dummies for whether the school contains a primary school, middle school, and high school. Demographic control variables are sourced at the county level (income, percent Hispanic, population) from IPUMs as described in the Data section. These schools are geolocated using ArcGIS.


\subsection{Controls and Other Non-Outcome Data}
Controls at the county level are sourced from IPUMS and consist of basic relevant demographic information: population, income, percent of county that is Hispanic etc. County level data is mapped to its relevant location using census data as well. 

Data on migration comes from the 2011-2015 American Community Survey (ACS), which reports the number of people moving from each origin county to destination county (aggregated over the four years).\footnote{ Historically, approximately 15\% of the ACS migration data has been allocated, or imputed based on salient characteristics (United States Census Bureau \cite{noauthor_american_2020}). } This sample also contains migration flows by Hispanic origin, allowing us to determine whether they move based on geographic boundaries.
% TODO: propagate errors based on the margin of error

Finally, data attached to specific outcomes are discussed under their relevant section.


\section{Empirical Strategy}\label{s:rd}

To isolate the causal effect of Spanish language television, I adapt the technique used in \cite{velez_tuning_2019}  and generalize it from two counties to the entirety of the US. Velez and Newman exploit a FCC (Federal Communications Commission) regulation which determines the distance from a TV station in which the station's broadcast signal is protected from interference.

Digital and satellite TV stations operate by broadcasting signals from a central antenna, and the field strength at a given point resulting from this antenna is a mechanical product of several factors: The antenna's ERP (Effective Radiated Power, which is the amount of input power given to the antenna adjusted for idiosyncrasies in the antenna that may boost or attenuate the effective power), the antenna's HAAT (High Above Average Terrain), and the distance from the point to the antenna.

%@CITE https://www.fcc.gov/media/radio/fm-and-tv-propagation-curves
%@CITE White Paper TAC

This signal declines in strength as one grows more distant from the station, making it subject to interference. The FCC regulation OET Bulletin No. 69 (FCC,\cite{noauthor_oet_2004}) protects signals for commercial TV stations from interference in a contour area for which service holds at 50\% of locations 90\% of the time.\footnote{ These contour lines are termed $(50,90)$ lines. There is a small adjustment made for different channel numbers, which have varying noise-limited coverage. } An example of this coverage contour can be seen in Figure \ref{contourexamplefig}; note that they tend to be sizable enough to fully cover major metropolitan areas, with contours boundaries ending substantially beyond them.

This creates a natural spatial regression discontinuity, where the decaying strength of a signal due to distance from a station is combined with this cutoff in broadcast protection to create a split among people just inside and outside these coverage contours that are presumably comparable save for their access to broadcast TV.  This minimizes the potential concern of omitted variable bias, as the groups we are comparing across this border should share many overarching characteristics.

In the case of Spanish Language TV in particular, this should allow us to examine its causal effect on Hispanic communities for spatially located outcomes. As mentioned, these contours are purely determined by an algorithm and only dependent on physical variables like local elevation and antennae strength. Thus, the precise regulatory boundaries are located in more or less random locations, and coverage is large enough that these contours tend to cut across towns and suburbs, rather than large cities --- television networks are not constructing their antennas to be just large enough to only cover the most dense and populous areas. This implies that network executives, if they are aiming to maximize profit, ratings, or audiences, would not have these boundaries at the forefront of their calculus.

In order for the causal effect of SLTV to be identified, the actual coverage of the contours must be uncorrelated with any of the other determinants for the outcome variables with which we are interested. One reassurance is that the interference protection regulation, OET Bulletin 69, was only codified in 1977 --- in contrast, Univision, the largest owner of SLTV stations, was founded in 1955, and had built a substantial number of their television stations and antennas by 1977.\footnote{ Though Telemundo, the second largest owner of SLTV, was technically founded in 1984, the stations it initially acquired were built in 1954. It also primarily expanded through the acquisition of existing stations, rather than building out its own new ones. } Furthermore, the most recent Longley-Rice methodology used to determine TV service coverage was only adopted in 1997, making it even less likely that stations were built or adapted in response to the policy.\footnote{ See FCC\cite{noauthor_oet_2004} for details.} Nonetheless, one may be concerned that SLTV stations target areas with more Hispanic people, or wealthier communities, or more populous areas, all of which are factors that could affect the areas of interest. Hence, I include explicit controls for these variables in the regression.
% @CITE univisino, telemundo

cite GREMBI 

The instrument therefore consists of two variables interacted: First, a dummy for whether the outcome data falls within a SLTV station's coverage contour boundaries, and second, the distance from the outcome of interest to the closest coverage boundary. To guarantee similarity between the people inside and outside the boundaries, only data points located within a distance of 100 KM of the boundary are kept.\footnote{ Using a round number in kilometers rather than miles makes the cutoff less likely to be correlated with some real-world phenomena.} 

Several concerns that potentially remain:
\begin{itemize}
\item \textit{Can we guarantee that it is Hispanic people who watch SLTV?} If it were the case that non-Hispanic people were frequent viewers of SLTV, the interpretation of the main effects would potentially be different: we would be looking at the effect of SLTV on all people. Thus, though outcomes restrict the analysis to how the lives of Hispanic people change, this could be driven by, for instance, white people treating Hispanic people differently due to having viewed SLTV. However, only 4\% of total SLTV station programming watched can be attributed to non-Hispanic people, a number that is only as high as it is because some SLTV stations also broadcast in English  (FCC\cite{noauthor_hispanic_2016}). Similarly, $<1\%$ of all programming watched by non-Hispanics is in the Spanish language.

\item \textit{How do we account for the possibility of selection?} It is theoretically possible that Hispanic people move in response to these television coverage contour boundaries, and that the effects seen are therefore a result of Hispanic people self-sorting. If this were true, it would be a fairly remarkable result---people moving in significant quantities for access to better television in a way that influences life outcomes ranging from education to business to politics. However, as the subsection on migration beneath demonstrates, the selection story does not appear to be borne out by the data.
\end{itemize}

\subsection{Main Specification}

A standard regression thus looks like restricting the universe of observations to only those within a small radius of the contour boundary, where the key independent variable of interest is an indicator for the observation being inside or outside the boundary, interacted with the distance to the boundary:
\[ Y_i^{} = \beta_0 + \beta \mathbb{I}[InsideContour_i] \times Distance_i + \gamma X_i + \epsilon_i \]

where $Y_i$ is an outcome for observation $i$ and $X$ is a vector of controls for the observation. The main coefficient of interest is $\beta$, and due to the nature of our instrument, we place the majority of interpretive weight on the indicator for being inside the television coverage contour. 

The unit of observation depends on the set of outcomes we are looking at. For firm data, we aggregate our data into a set of grid points (typically roughly 2 $\times$ 2 KM in size) so that we can compare the number of firms across areas.\footnote{ In addition to providing cleaner interpretability, grouping data into `raster' form is also less computationally intensive for the analysis.} For school data, the unit of observation is a single school, as we have school-level controls. I typically aggregate into grids by taking the sum of observations within grids (i.e., the number of Hispanic-owned businesses within a grid point), except where otherwise noted.

I prefer to leave standard errors robust, and separately check for robustness with respect to spatial autocorrelation for each main result. Other fixed effects/clustering options are treated similarly.

\subsection{Spatial Autocorrelation}

Spatial autocorrelation, or spatial dependence, occurs when our outcomes of interest are correlated with itself in space \citep{cliff_spatial_1973}. In general, this only means that I allow for $Cov(Y_i,Y_j) \neq 0$ when $i\neq j$ for locations $i$, $j$. For tractability, when given a dataset with $n$ locations, I place more structure on the problem, constructing a $n \times n$ spatial weights matrix $W$ with entries $w_{ij} = 1$ if locations $i,j$ are considered neighbors, and $w_{ij} = 0$ otherwise \citep{anselin_spatial_1998}. For data that takes the form of grids in space, I construct weights based on the rook criterion (grid points have unit weight if they share an edge), while for points in space, I assign unit weight to the four nearest neighbors as a comparable measure.

There are two primary models of spatial autocorrelation that I conduct robustness tests for:

\paragraph{The Spatial Autoregressive Model}

In this model, the spatial autocorrelation is directly accounted for in the specification:
\[ Y = \beta_0 + \rho W Y + \beta \mathbb{I}[InsideContour] \times Distance + \gamma X + \epsilon \]

This model is identical to the prior main specification, except for the addition of the $\rho W Y$ term, where $W$ is the aforementioned spatial weights matrix, and $\rho$ the autoregressive coefficient. In this model, spatial dependence affects the outcome variable only (e.g. a Hispanic business opening up may induce neighbors to not open businesses, due to an increased amount of competition in the area).

\paragraph{The Spatial Error Model}

In this model, the autocorrelation occurs in the error term:
\[ Y = \beta_0 + \beta \mathbb{I}[InsideContour] \times Distance + \gamma X + \epsilon \]
\[\epsilon = \lambda W \epsilon + \nu\]

This model is identical to the main specification, except the error terms are now additionally correlated due to the addition of the $\lambda W \epsilon$ term. In this model, spatial dependence enters through  the presence of missing spatial covariates which may affect the outcome.\footnote{ In particular, this allows us to further adjust for unique features of Hispanic communities, such as the geographic clustering of immigrants as \cite{cutler_when_2008} and \cite{cascio_cracks_2012} find.}

\subsection{Migration}

While it is theoretically conceivable that Hispanics would move based on access to SLTV, causing results to be driven by selection and confounding the direct effect of television itself, I demonstrate that movement across these coverage contours is minimal.

As mentioned in Section \ref{secdata}, the migration data from the ACS is provided at the origin county-destination county level. Given the relative size of a county, to define whether a county is inside a coverage contour or not, we further impose that at least 95\% of the area that the county encompasses must be inside of the coverage contour.\footnote{ Results are robust to different area cut-offs for a county to be considered inside the coverage contour.} We present summary statistics for this sample in Table \ref{sum_init}.

Tables \ref{mig_orig} and \ref{mig_dest} present the results on migration. These tables present results at the origin county - destination county level, tracking the Inverse Hyperbolic Sine (IHS) transformed values of the number of Hispanic migrants between the two counties.\footnote{ The IHS transform can be interpreted similarly to the Log transform (\% changes), but has the added advantage of being able to handle cases when $0$ is the observed value.} Table \ref{mig_orig} restricts to only origin counties that are within 100 KM of a coverage contour (the standard cut-off distance used for later outcomes).\footnote{ There are 636 such counties. The average origin county has 20 destination counties for which there is significant enough cross-county Hispanic migration that the ACS reports data for it.} In panel A, this is further restricted to origin counties inside the television contour, and so the main variable of interest is the dummy for the destination county being outside the TV contour. We observe a clear negative and significant relationship for migrations that cross the coverage contour. We interact the distance to the origin/destination with the TV dummy to ensure that we are controlling for all distance related effects, and control for county level characteristics including Log Population, Log Income, and percent of the county that is Hispanic for both origin and destination. All specifications also include origin fixed effects.

In panel B, we restrict to origin counties outside the television contour, and the main variable of interest is the dummy for the destination county being \textit{inside} the TV contour. In this case, the point estimate is negative, although results are overall insignificant---this is sufficient for us to make our argument, given that so long as there are not positive coefficients, there is no evidence of migration across contour borders.

Table \ref{mig_dest} repeats the analysis, this time restricting to only destination counties within 100 KM of a coverage contour. Results closely echo those seen in the prior table, with negative coefficients associated with migration across coverage contours, significant when the destination is inside the contour and not when they are outside. 

These results combined indicate that movement across coverage contours is not a major threat to identification. Even in cases where insignificant results are observed, the base rate of migration is not very high to begin with---in our origin county sample, an average of 84 Hispanic people are observed to move between each county-county pair (median: 25) over the five year period which the dataset spans. This also speaks to the magnitude of the coefficients observed, where the drop in 10 to 40\% of migrants observed still falls within a plausible range. Though we do not have theories as to why people may be \textit{averse} to moving across coverage contour boundaries, it is in and of itself an interesting result perhaps worth further investigation.

% include 0s

\subsection{Evidence of a first stage: do Hispanics in SLTV coverage contours watch more television?}

Over 85\% of Hispanic households own a television. A further 85\% of SLTV viewership occurs over satellite or broadcast television---important because coverage contours are only applicable for these types of television.\footnote{See \cite{noauthor_hispanic_2016} and \cite{de_la_merced_att_2014}. This fraction of broadcast TV viewership is substantially larger than the 24\% national average.} Thus, there is a substantial population that may be affected by the presence of these coverage contours.
%  Of these, over half get their television content via satellite or broadcast television---

To test for viewership 

American Time Use Survey

% Second, Hispanics consume substantial amounts of television---out of the 115 million households with television in the United States, there are 14 million Hispanic ones, proportional to the overall fraction of households. 

Compared to other viewers of television, Hispanics are also uniquely likely to watch television in a social context rather than watching alone---this is partially driven by the fact that non-Hispanic households have 40\% more TV sets per person than Hispanic ones \citep{coghill_tuning_2018}. This social aspect, wherein SLTV is watched with family and friends, may be one way in which identity is reinforced through television. 
% https://www.effectv.com/blog/tuning-hispanic-audiences
% even the largest markets do not locally source more than 20\% of their Spanish language programming - from foreign countries or domestic



%%%%%%%%%% PUBLIC SCHOOLS %%%%%%%%
\section{The impact of Spanish language television on Hispanic educational performance}\label{s:school}

In this section, we examine the performance of Hispanics in public schools and find that while academic achievement generally increases and disciplinary issues generally decrease in response to SLTV, the opposite holds true when the measures are more directly to identity. 


\subsection{Results}

Table \ref{edu_top} presents the standard specification for the education dataset, looking at the effect of television on schools within 100 KM of a coverage contour. For each of these measures of academic achievement, column (1) includes only county level controls, column (2) adds controls for school size (number of students and teachers), and column (3) adds controls for whether the school contains primary/middle/high school divisions. Panel A examines the effect of television on the IHS of the number Hispanic students considered gifted, while panels B and C look at the effect on the number of Hispanic students enrolled in an AP course or passing at least one AP course respectively. The coefficient of interest, the dummy for whether the school is located within a coverage contour or not, is significant at the 5\% level for all columns and panels. The effect sizes are modest, but non-trivial: an approximately 1.5\% increase in the number of gifted students, and increases on the order of roughly 5\% for the number of students involved in Advanced Placement curricula. 

Table \ref{edu_bot} examines the effect of SLTV on disciplinary incidents: Panel A presents the effect on the number of Hispanic students ever given an out of school suspension over the prior school year, while Panel B presents this on the number of Hispanic students considered chronically absent. These results are all significant at the 1\% level for all columns and panels. The effect sizes are comparable to that regarding academic achievement, displaying a 1.5\% decrease in the number of students suspended, and a 7\% decrease in the number of students who are chronically absent.

Table \ref{edu_mech} examines the effect of SLTV on outcomes more directly tied to identity: Panel A presents the effect on the number of students categorized as having Limited English Proficiency. These effects are significant at the 1\% level, and represent a 3-4\% increase in the number of students designated under this category. Panel B, on the other hand, presents the effect on the number of Hispanic students who are ever victims of harassment on the basis of their ethnicity. These results are significant at the 10\% and 5\% levels, and account for a small .2\% bump in the number of such cases. 

\paragraph{Robustness} To test the robustness of these results, we present Table \ref{edu_ap_robust}, which uses as its outcome variable the number of Hispanic students passing the AP. We choose to present robustness on this outcome in particular due to its lower sample size --- it is a priori the most likely to be underpowered. Column (1) presents the baseline results (it is identical to column (3) of Table \ref{edu_top}), while column (2) includes the interaction of the TV dummy with the distance to the boundary squared. This is plausibly relevant to the main effect, given that television signals decay in strength in proportion to the square of the distance. Columns (3) and (6) reduce the cutoff distance from the boundary to one half and one third of the original 100 KM limit. Column (4) includes county level fixed effects. Column (5) additionally controls for the total number of APs passed by all students. The robustness checks hold up across the board with all columns maintaining significance, although the 33 KM boundary limit is close to underpowered. Robustness checks on the other outcome variables of interest hold up to a similar analysis.

Finally, we may be concerned about the potential effects of spatial autocorrelation in the data. A Moran's I test using 4 nearest neighbours between the schools indicate that there is spatial autocorrelation at any reasonable level for $\alpha$. Hence, Table \ref{edu_spatial} presents two alternate models that control for the effects of spatial autocorrelation. Column (2) uses a spatially autoregressive lag model, wherein the outcome variable may be correlated with its neighbours. Column (3) uses a spatially autoregressive error model, wherein the presence of missing spatial covariates (causing correlated errors) is adjusted for. In both cases, the alternate models yield results that closely resemble the standard specification in column (1). 


\subsection{Discussion}

\paragraph{Evidence of Identity as a Mechanism}
The results in Table \ref{edu_mech} provide some concrete evidence that identity changes as a result of the effect of television. We believe that access to SLTV reinforces Hispanic identities, making them more salient to the Hispanic individuals consuming the broadcast programs. The most direct evidence for this stems from the results on harassment and bullying based on ethnicity. Given that very few non-Hispanic people view SLTV programming, the fact that more Hispanic students are bullied on the basis of their ethnicity suggests that some change must be occurring within the students along this dimension.\footnote{ This increase in bullying does not appear to be the result of `retaliation' to Hispanic students bullying others: the coefficient only attenuates slightly when further controlling for the total number of students bullied, and running the main specification with the number of Hispanic students as perpetrators of race/ethnicity based bullying yields an insignificant negative coefficient. } 

A substantial literature has shown that increased visibility of (non-majority) ethnicities is associated with greater amounts of bullying,\footnote{ See \cite{scherr_bullying_2009} for a review of this literature.} consistent with the results that we see. Though it is impossible to rule out all other stories (perhaps children who watch more TV overall are more likely to be victims of bullying---but this is not supported by the literature. If anything, there is support for television causing children to become bullies \citep{kuntsche_television_2006}, but this is not borne out in our data), the most parsimonious explanation is one in which television increases identity salience and hence ethnicity-based bullying.

We make a similar argument in the interpretation of the greater number of Hispanic students classified as having Limited English Proficiency. This increase demonstrates that these students possess a lower degree of command over the English language, suggesting two possibilities: either (1) that academic/linguistic abilities are lowered across the board, or that (2) there is some substitution in ability towards the Spanish language instead. Given that academic abilities appear to be \textit{enhanced} by the presence of SLTV, the substitution story appears more plausible to us.\footnote{ Granted, the measures of academic ability measure only the performance of students at the top end. But given the existence of these results, a countervailing narrative in which SLTV decreases the academic performance for other Hispanic students would need require a mechanism that could produce such differential effects.} Unfortunately, we do not have direct evidence on the Spanish-speaking abilities of students, and so recognize that this is not a settled matter. Thus, while the evidence presented is fairly suggestive, more research could be done on this matter.

\paragraph{Effects on Academic Achievement and Discipline} 
We next turn our attention to the results presented in Tables \ref{edu_top} and \ref{edu_bot}. The results on academic achievement unambiguously show that, for the top end of Hispanic students, performance is bolstered by the presence of SLTV. This effect appears to hold across students of all ages --- while gifted programs are typically aimed at students in primary and middle schools, AP courses and exams are almost exclusive taken by high schoolers. 

 The number of observations recorded for these regressions is worth addressing: compared to the 40,000 schools seen in other regressions, there are only 26,000 seen for gifted students, and fewer than 10,000 for the AP results. In the case of gifted programs, this drop is due to the fact that schools which do not have gifted student programs were omitted from the sample. We find it unlikely that the presence of a gifted program in a school is correlated with the the school being placed just inside or outside a television coverage contour, and so do not believe that this omission introduces any bias. Similarly, in the case of the AP results, only 9,765 of the schools in the sample are high schools with 12th graders enrolled in them---hence, the observed 6,089 schools opting to self-report AP course results is still sizable. Though the number of schools reporting AP exam results is substantially lower and may be concerning for this result, this can at least partially be attributed to the fact that students directly receive their AP scores, and the schools at which they are enrolled may not always have access to their AP scores. Furthermore, given that overall AP scores do not meaningfully change, it is unlikely that there is substantial selection into score reporting over the concerns of Hispanic students passing AP scores---especially because there are no real-world incentives or benefits attached to doing so.

Noting that increases in AP enrolment are predictive of higher rates of college enrolment and degree attainment \citep{speroni_determinants_2011}, it is likely that SLTV can have downstream effects beyond simply greater academic attainment in the short term. Running counter to the mainstream narrative, these increases in academic performance match the results found by \cite{gentzkow_preschool_2008}, who find that television increases test scores for preschoolers (and in particular, preschoolers from households where English is not the dominant language).

Similarly, these increases in disciplinary outcomes can ameliorate the serious downstream effects that exist beyond the disciplinary event itself: the literature suggests that not only are suspended students at immediate risk of academic harm and further disciplinary issues \citep{arcia_achievement_2006}, but that these students are also more likely to be incarcerated as adults  \citep{wolf_school_2017}. Non-disciplined students appear to suffer from spillover effects in their academic performance as well \citep{perry_suspending_2014}.
% HOW MANY STUDENTS get bullied? 7169 cases, so 22 students


% TODO: robustness on distance, spatial error, weighting by # students, imputing missings, 

On the whole, this suggests that the lives of Hispanic students living may materially improve along academic and social dimensions as a result of SLTV. 

\paragraph{The Difference Between `Identity' and Other Outcomes}
It appears that while Hispanic discipline issues are generally improved by SLTV, this does not extend to the measure directly tied to identity: bullying and harassment based on ethnicity. Similarly, while academic achievement is generally improved by SLTV, this finding does not also generalize to LEP rates. This puzzle---explaining how identity driven results move in opposite directions from the others---is not easily resolved without relying on identity in some form. 

Though we do not have a rigorous argument that can fully resolve this puzzle, we suggest that a substitution effect based on SLTV affecting identity can explain much of the results seen. That is, SLTV might in the immediate affect the identity based mechanisms that we see (more social issues, worse academic performance on metrics tied to identity), but that student performance in other non-identity tied outcomes might in turn shift to make up for the difference. If this were the case, we would expect to see results in line with what we see. An alternative explanation not relying on identity would still need to be able to explain why most academic and disciplinary measures point in one direction, whereas the ones more tightly linked to identity reverse.



\section{The identity mechanism}\label{s:mech}

SLTV programming is also more likely to contain content that is directly salient to a Hispanic person's identity. This occurs not only because of the language of the broadcast, but also its content: roughly 20\% of programming on SLTVs are telenovelas produced in foreign (Latin American) countries, with a similar proportion of programming dedicated to non-locally produced news and paid programming.\footnote{For more information, see \cite{noauthor_hispanic_2016}. }


We marshal three



%%%%%%%%%% safegraph %%%%%%%%
\subsection{Safegraph foot-traffic data}\label{s:safegraph}



\paragraph{Firm Name Classification} Unlike the names of firm principals, there is no readily available or standardized method to determine whether a firm's name is a `Hispanic name' or not. Although a machine learning approach is still theoretically possible under these circumstances, a quick visual inspection of the data revealed that a relatively low percentage of firms had names that were explicit tied to a Hispanic identity---hence, many approaches would likely identify a significant number of false positives.

In order to be conservative and ensure that firms identified as bearing Hispanic names actually are such, we construct a measure that classifies a firm name as Hispanic if it contains certain keywords that are explicitly associated with a Hispanic identity. These keywords are split into three major categories: (1) References to countries in Latin America or Latin America itself. Firms that include the base forms of country names in Latin America are considered to be explicitly referencing a Hispanic identity (examples include: `Cuban Guys 102, LLC' , `Bravo Latino Brands, LLC.') (2) Names containing common one of the top 50 most Spanish words that are not also present in English (examples include: `La Joya Estates, Ltd.', `Conselho Nacional De Saude Mental E Medicina Psicossomatica Inc.'), and (3) Names containing common Hispanic foods. (examples include: `Charlie Cactus Tacos, LLC', `Taqueria Casas 2 Inc.') Due to the lack of a systemic means to construct this last category, I conduct robustness checks dropping this category; results do not substantially change when omitting this third category. Table \ref{firm_class} contains a list of these keywords as well as some additional detail on the classification process.

Out of our sample, 1.1\% of firms meet this criteria (1\% if omitting the food-based names). A manual check of firms that are classified as Hispanic also confirms that the firm name classification process succeeds.

% cluster: Esarey (2019) following Ibragimov and Muller 2010




\section{Conclusion} \label{s:conclusion}

In this paper, we provide a number of high-level results: we show that SLTV has a substantial impact on reducing the inequality that Hispanics face as entrepreneurs and students. From a business standpoint, SLTV increases Hispanic firm ownership, while also increasing the total number of firms bearing Hispanic names, pointing to an expansion in demand for goods and services linked to the Hispanic identity. From an educational standpoint, SLTV further improves the academic performance of top achievers while decreasing the occurrence of disciplinary issues among Hispanic students; instances in which this is violated are instances that one would expect to arise from a stronger sense of identity being reinforced. 

Undergirding these findings is the consistent notion that identity is strengthened from the presence of television, and though we cannot ever perfectly confirm that this is the case, we believe the cumulative weight of the results is suggestive. However, this would be a prime area for future work to be done: we think that a more precise and direct effect on identity from the media (for both Hispanics and other minorities), would be of value---especially if one could show its relative influence and power over time.



Political consequences?
Not a monolith, what about subgroups? How do these interact with other dimensions along which the achievement gap has been observed, such as socioeconomic status or gender?

More broadly speaking, we think that looking at the spillover effects of identity for both those within the in-group and those in the out-group would be of interest (how are Hispanic people who don't watch SLTV be affected by peers who do? How do white people, or other minorities react?). Finally, it may also be interesting to examine the role that media as a whole plays on identity, and whether Spanish Language Television serves as a complement or substitute with other forms of media.


\clearpage
\pagebreak

\begin{singlespace}
\begin{scriptsize}
\bibliographystyle{aea}
%\bibliographystyle{plainnat}
\bibliography{tv}{}
\end{scriptsize}
\end{singlespace}

\pagebreak
\clearpage

\section*{Figures and Tables}

%\subsection{Figures}

%%%%%%%%%%%%%%%%%%%%%%%%%%%%%%%%%%
%% Figure 1: Coverage Contours
%%%%%%%%%%%%%%%%%%%%%%%%%%%%%%%%%%
%
%\begin{figure}[!hbtp]
%\centering
%\caption{Dummy for Hispanic Owned Business with Hispanic Name by Distance to Contour Boundary }\label{busnnnamefig}
%\includegraphics[width=12cm]{../../analysis/Output/graphs/hispanicbusnname.pdf}
%
%\textit{Notes:} The figure presents data at the firm level, where a smoothed average of a residualized dummy for Hispanic businesses with Hispanic-indicating names is plotted against the distance of the school to the closest Spanish Language Television station contour boundary. Positive distances denote schools that are located within the boundary, while negative distances denote schools outside of them. Controls at the county level include log population, income, and percentage population Hispanic.
%\end{figure} 
%
%\begin{figure}[!hbtp]
%\centering
%\caption{IHS(\# Hispanic Students Suspended) by Distance to Contour Boundary }\label{suspensionsfig}
%\includegraphics[width=12cm]{../../analysis/Output/graphs/hispanicsuspensions.pdf}
%
%\textit{Notes:} The figure presents data at a school level, where a smoothed average of the inverse hyperbolic sine transformed counts of Hispanic students suspended is plotted against the distance of the school to the closest Spanish Language Television station contour boundary. Positive distances denote schools that are located within the boundary, while negative distances denote schools outside of them.
%\end{figure} 

%\begin{figure}[!hbtp]
%\centering
%\caption{IHS(\# Hispanic Donations to Trump) by Distance to Contour Boundary }\label{donationsfig}
%\includegraphics[width=12cm]{../../analysis/Output/graphs/hispanictrump.pdf}
%
%\textit{Notes:} The figure presents data aggregated into squares of size approximately 4 KM$^2$, where a smoothed average of the inverse hyperbolic sine transformed counts of Hispanic campaign contributions to Trump for the 2016 election is plotted against the distance of the school to the closest Spanish Language Television station contour boundary. Positive distances denote schools that are located within the boundary, while negative distances denote schools outside of them.
%\end{figure} 

\begin{figure}[!hbtp]
\centering
\caption{Coverage Map for TV Station WUVC-DT}\label{contourexamplefig}
\includegraphics[width=15cm]{../../analysis/Output/img/ContourExample.png}
\end{figure} 

\begin{figure}[!hbtp]
\centering
\caption{The Coverage Contours of Spanish Language TV stations}\label{contourfig}
\includegraphics[width=8cm]{../../analysis/Output/img/SpanishContours.png}
\end{figure} 

\begin{figure}[!hbtp]
\centering
\caption{Map of School Districts in the US}\label{schooldistrictfig}
\includegraphics[width=12cm]{../../analysis/Output/img/LEAMap.png}
\end{figure} 

\clearpage

%\subsection{Tables}
%
%Table plan:
%\begin{enumerate}
%\item $<X>$ Summary stats table: ATUS data, education (school level), transcript data, Safegraph data
%\item $<X>$ ATUS: first stage (93), children (99), ATUS: parents (97),
%\item $<X>$ Education: top performance: gifted, SAT/ACT, AP passed, 
%\item $<X> $Education: identity
%\item $<X>$ Transcript: identity
%\item $<X>$ Safegraph: identity
%\end{enumerate}
%
%Appendix:
%\begin{enumerate}
%\item Migration table
%\item $<X>$ foreign born (94)
%\item Education: robustness
%\item Education: robustness spatial (SAR lag, SAR error)
%\item $<X>$ Education: more top performance
%\item Education: bottom performance?
%\item Transcript: bad identity
%\item Additional robustness
%\end{enumerate}

\expandableinput{../../analysis/Output/regs/summary_main.tex}
\expandableinput{../../analysis/Output/regs/atus_main.tex}
\expandableinput{../../analysis/Output/regs/edu_main.tex}
\expandableinput{../../analysis/Output/regs/edu_mech.tex}
\expandableinput{../../analysis/Output/regs/transcript_main.tex}
\expandableinput{../../analysis/Output/regs/safegraph_main.tex}


%%%%%%%%%%%%%%%%%%%%%%
%%%%%%%%%%%%%%%%%%%%%%

%ONLINE APPENDIX MATERIAL


\clearpage

\singlespacing

\setcounter{footnote}{0}

\setcounter{section}{0}

\setcounter{page}{1}
\renewcommand\thepage{A.\arabic{page}}

\renewcommand*{\theHsection}{\arabic{section}.\arabic{section}} 

\renewcommand*{\theHfigure}{\arabic{section}.\arabic{figure}} 
\setcounter{figure}{0}
\renewcommand\thefigure{A.\arabic{figure}}


\renewcommand*{\theHtable}{\arabic{section}.\arabic{table}} 
\setcounter{table}{0}
\renewcommand\thetable{A.\arabic{table}}

\renewcommand{\thesection}{Appendix \Alph{section}}



\begin{center}
\Large ONLINE APPENDIX
\end{center}

\section{Auxiliary data sources} \label{a:auxiliarydata}

In addition to the primary data sources described in Section~\ref{s:data}, we also use a number of auxiliary data sources for the empirical analysis.


\paragraph{American Time Use (ATUS)}

\paragraph{Civil Rights Data Collection (CRDC)}

The outcome data from the CRDC can be split into two categories:
\begin{itemize}
\item \textbf{Academic Achievements:} We focus on two outcomes that track the effect of television on the top end of the academic distribution of students: the number of Advanced Placement (AP) classes students enrol in and pass, as well as the number of students placed into gifted programs, and one outcome on the bottom: the number of students with Limited English Proficiency (LEP).

The AP program is administered by the College Board, and defines a standardized college-level curriculum that is taught to high school students in AP Classes. In conjunction with AP Classes, AP Exams are national examinations which are designed to test mastery of material taught in AP classes. These exams are scored on a scale ranging from 1 to 5, with scores below a 3 marked as a failed exam. Even among the students who select into these classes (22\% in 2015\footnote{ Data computed from number of high school graduates in 2015 (National Student Clearinghouse Research Center,\cite{noauthor_high_2015}), and number of seniors who sat an AP exam in 2015. This is how the College Board currently tracks national AP participation (no comparable summary statistic was released in 2015) (College Board,\cite{noauthor_ap_2015})}), a substantial number of students who take these exams fail them - approximately 35\% (College Board\cite{noauthor_ap_2020}). 

Gifted and talented programs are ``programs during regular school hours that provide special educational opportunities including accelerated promotion through grades and classes and an enriched curriculum for students who are endowed with a high degree of mental ability or who demonstrate unusual physical coordination, creativity, interest, or talent." (CRDC\cite{noauthor_master_2016}) These programs, while not mandatory, are common across school districts, and vary in their implementation. % HOW MANY STUDENTS?

LEP students (also called English Learner students) are students that, as a result of their limited command over the English language, have difficulty participating in regular school activities.\footnote{The specific definition of a LEP student depends on individual state regulation, but must also satisfy the criteria outlined under Title IX of the Elementary and Secondary Education Act (US Department of Education,\cite{noauthor_elementary_2004}). The most salient features of Title IX are that students must either not speak English as a native language or come from an environment where non-English languages are dominant, and also face substantial difficulty in engaging with others on the basis of their English ability.} 9\% of all public school students are considered LEP, and while students are placed into the program is at the discretion of individual school districts, all districts must provide language assistance services and have staff qualified to implement the LEP programs.\footnote{ Department of Justice and Department of Education,\cite{noauthor_ensuring_2015} contains a full enumeration of the responsibilities school districts have. It further includes requirements such as ensuring equal access to various school programs etc. } 

\item \textbf{Disciplinary Issues:} Three forms of academic discipline are considered as outcome variables: the number of out of school suspensions, the number of absences, and the amount of harassment and bullying on the basis of race/ethnicity experienced by students.

Out of school suspensions are instances ``in which a child is temporarily removed from his/her regular school for at least half a day (but less than the remainder of the school year) for disciplinary purposes to another setting (e.g., home, behavior center)." (CRDC,\cite{noauthor_master_2016}) We consider only students without disabilities, and note that depending on school policy, educational services may still be provided during this time.\footnote{Students with disabilities served under IDEA face substantially different suspension policy.}

A chronically absent student is one ``who is absent 15 or more school days during the school year. A student is absent if he or she is not physically on school grounds and is not participating in instruction or instruction-related activities at an approved off-grounds location for at least half the school day." (CRDC,\cite{noauthor_master_2016}) Each day for which a student is absent for 50 percent or more of the school day is counted. Absences are counted regardless of whether they are excused or not, and so include absences due to illness, needing to care for a family member, or simple truancy.

Harassment or bullying on the basis of race, color, or national origin ``refers to intimidation or abusive behavior toward a student based on actual or perceived race, color, or national origin. Harassing conduct may take many forms, including verbal acts and name-calling, as well as non-verbal behavior, such as graphic and written statements, or conduct that is physically threatening, harmful or humiliating. The conduct can be carried out by school employees, other students, and non-employee third parties. Bullying on the basis of race, color, or national origin constitutes racial harassment." (CRDC,\cite{noauthor_master_2016}) Though there are other categories of bullying and harassment reported (and other types of infractions and disciplinary measures taken), these are less directly relevant to the question at hand.


\end{itemize}

Notably, all the outcome information described above is also provided for Hispanic subpopulations --- hence, the outcome of interest is generally the number of Hispanic students passing AP tests, or being bullied on the basis of their ethnicity, etc. These variables are all reported at the school level. 



\clearpage

\section{Additional figures and tables}

\expandableinput{../../analysis/Output/regs/atus_foreign.tex}

\expandableinput{../../analysis/Output/regs/edu_extra_achieve.tex}




%
% Table created by stargazer v.5.2.2 by Marek Hlavac, Harvard University. E-mail: hlavac at fas.harvard.edu
% Date and time: Fri, Dec 13, 2019 - 20:45:59
\begin{table}[!htbp] \centering 
  \caption{School-District Level Summary Statistics} 
  \label{} 
\begin{tabular}{@{\extracolsep{5pt}}lccccccc} 
\\[-1.8ex]\hline 
\hline \\[-1.8ex] 
Statistic & \multicolumn{1}{c}{N} & \multicolumn{1}{c}{Mean} & \multicolumn{1}{c}{St. Dev.} & \multicolumn{1}{c}{Min} & \multicolumn{1}{c}{Pctl(25)} & \multicolumn{1}{c}{Pctl(75)} & \multicolumn{1}{c}{Max} \\ 
\hline \\[-1.8ex] 
Distance to Boundary & 17,280 & 136.855 & 146.751 & 0.000 & 15.786 & 217.567 & 806.543 \\ 
SLTV Coverage Dummy & 17,280 & 0.292 & 0.455 & 0.000 & 0.000 & 1.000 & 1.000 \\ 
\% County Hispanic & 17,280 & 7.051 & 11.950 & 0.000 & 0.668 & 6.974 & 97.216 \\ 
Log(Population) & 17,280 & 11.618 & 1.840 & 5.869 & 10.242 & 13.110 & 15.997 \\ 
Log(Income) & 17,280 & 9.428 & 0.257 & 7.976 & 9.257 & 9.593 & 10.245 \\ 
\hline \\[-1.8ex] 
\multicolumn{8}{l}{\textit{Note:} Distance to SLTV Boundary measured in KM} \\ 
\end{tabular} 
\end{table} 

%
% Table created by stargazer v.5.2.2 by Marek Hlavac, Harvard University. E-mail: hlavac at fas.harvard.edu
% Date and time: Fri, Dec 13, 2019 - 22:32:28
\begin{table}[!htbp] \centering 
  \caption{School Level Summary Statistics} 
  \label{} 
\begin{tabular}{@{\extracolsep{5pt}}lccccccc} 
\\[-1.8ex]\hline 
\hline \\[-1.8ex] 
Statistic & \multicolumn{1}{c}{N} & \multicolumn{1}{c}{Mean} & \multicolumn{1}{c}{St. Dev.} & \multicolumn{1}{c}{Min} & \multicolumn{1}{c}{Pctl(25)} & \multicolumn{1}{c}{Pctl(75)} & \multicolumn{1}{c}{Max} \\ 
\hline \\[-1.8ex] 
Total Students & 96,349 & 524.859 & 449.354 & 2.000 & 254.000 & 662.000 & 14,164.000 \\ 
\# Hispanic Students & 91,019 & 143.195 & 243.873 & 2.000 & 13.000 & 166.000 & 7,675.000 \\ 
Contains Grade 1 & 96,350 & 0.538 & 0.499 & 0 & 0 & 1 & 1 \\ 
Contains Grade 6 & 96,350 & 0.364 & 0.481 & 0 & 0 & 1 & 1 \\ 
Contains Grade 9 & 96,350 & 0.253 & 0.435 & 0 & 0 & 1 & 1 \\ 
Hispanic Suspension Dummy & 94,535 & 0.382 & 0.486 & 0.000 & 0.000 & 1.000 & 1.000 \\ 
Hispanic Chronic Absentees & 94,540 & 22.920 & 57.838 & 0.000 & 0.000 & 22.000 & 2,131.000 \\ 
\# Teachers & 93,934 & 35.219 & 33.892 & 1.000 & 19.000 & 44.000 & 6,031.000 \\ 
\hline \\[-1.8ex] 
\multicolumn{8}{l}{\textit{Note:} Dummies indicate whether event occurred in the school over the past year} \\ 
\end{tabular} 
\end{table} 

%\begin{table}[!h]
	\centering
	\captionsetup{skip=1.5pt}
	\caption{Summary Statistics}\label{sum_init}
	\scalebox{.8}{
		\begin{threeparttable}
			\begin{tabular}{l@{\extracolsep{4pt}}cccc}
				\hline\hline\addlinespace
				& \textit{All} &  \textit{No TV} & \textit{TV} \\
				\cline{2-4} \addlinespace
				& (1) & (2) & (3) \\
				\hline\addlinespace Panel A: Firms & \multicolumn{3}{c}{} \\
				\hline\addlinespace
				IHS(Hispanic Owned Firms) & 0.992 & 0.671 & 1.225 \\
				& (1.694) & (1.308) & (1.892) \\
				Hispanic Named Firms & 0.027  & 0.006 & 0.042 \\
				& (0.161) & (0.080) & (0.200) \\
				Log Income & 9.498 & 9.463 & 9.523 \\
				& (0.241) & (0.284) & (0.201) \\
				Log Population & 11.954 & 11.206 & 12.497 \\
				& (1.398) & (1.253) & (1.239) \\
				\% County Hispanic & 0.086 & 0.063 & 0.103 \\
				& (0.105) & (0.061) & (0.125) \\
				\hline\addlinespace
				\hline\addlinespace
				Panel B: Migrations & \multicolumn{3}{c}{} \\
				\hline\addlinespace
				IHS(Hispanic Migrants) & 4.331 & 4.035 & 4.462 \\
				& 1.297 & 1.204 & 1.316 \\
				Log Income & 9.513 & 9.449 & 9.541 \\
				& (0.281) & (0.201) & (0.305) \\
				Log Population & 12.358 & 11.942 & 12.542 \\
				& (1.516) & (1.640) & (1.419) \\
				Fraction County Hispanic & 0.124 & 0.085 & 0.141 \\
				& (0.155) & (0.122) & (0.164) \\
				\hline\addlinespace
				 \hline\addlinespace
			\end{tabular}
			\begin{tablenotes}[flushleft]
				\item \textit{Notes:} The table presents means (and standard deviations). Variables in Panel A and C aggregate data from firms and campaign contributionsn into 2 KM$^2$ grid points in Florida and the USA respectively. Variables in Panel B refer to our schools sample.  Column 1 shows data for all observations.  Columns 2 and 3 show data for the subsample without and with SLTV coverage, respectively. All panels only keep observations within 100 KM of the coverage contour. No control is significantly different across the coverage contour at the $\alpha = .1$ level.
			\end{tablenotes}
		\end{threeparttable}
	}
\end{table}

\begin{table}[!h]
	\centering
	\captionsetup{skip=1.5pt}
	\caption{Summary Statistics}\label{sum_out}
	\scalebox{.68}{
		\begin{threeparttable}
			\begin{tabular}{l@{\extracolsep{4pt}}cccc}
				\hline\hline\addlinespace
				& \textit{All} &  \textit{No TV} & \textit{TV} \\
				\cline{2-4} \addlinespace
				& (1) & (2) & (3) \\
				\hline\addlinespace Panel A: Firms & \multicolumn{3}{c}{} \\
				\hline\addlinespace
				IHS(Hispanic Owned Firms) & 0.992 & 0.671 & 1.225 \\
				& (1.694) & (1.308) & (1.892) \\
				Hispanic Named Firms & 0.027  & 0.006 & 0.042 \\
				& (0.161) & (0.080) & (0.200) \\
				Log Income & 9.498 & 9.463 & 9.523 \\
				& (0.241) & (0.284) & (0.201) \\
				Log Population & 11.954 & 11.206 & 12.497 \\
				& (1.398) & (1.253) & (1.239) \\
				Fraction County Hispanic & 0.086 & 0.063 & 0.103 \\
				& (0.105) & (0.061) & (0.125) \\
				Observations & 23,823 & 10,023 & 13,830\\
				\hline\addlinespace
				\hline\addlinespace
				Panel B: Schools & \multicolumn{3}{c}{} \\
				\hline\addlinespace
				IHS(Hispanic Gifted Students) & 1.988 & 1.262 & 2.380 \\
				& (1.552) & (1.238) & (1.563) \\
				IHS(Hispanic AP Enrolment) & 3.192 & 2.091 & 3.778 \\
				& (1.937) & (0.646) & (0.918) \\
				IHS(Hispanic AP Passes) & 4.087 & 3.497 & 4.181 \\
				 & (0.917) & (0.646) & (0.918) \\
				 IHS(Hispanic Suspensions) & 0.957 & 0.676 & 1.102 \\
				 & (1.273) & (1.044) & (1.353) \\
				 IHS(Hispanic Absentees) & 2.655 & 1.881 & 3.054 \\
				 & (1.765) & (1.536) & (1.742) \\
				 IHS(Hispanic Limited English Proficiency) & 2.915 & 2.113 & 3.331 \\
				 & (2.040) & (1.820) & (2.024) \\
				 IHS(Hispanic Harassment) & 0.045 & 0.027 & 0.055 \\
				 & (0.273) & (0.211) & (0.299) \\
				 Log Income & 9.547 & 9.430 & 9.608 \\
				 & (0.303) & (0.200) & (0.328) \\
				 Log Population & 12.484 & 11.559 & 12.964 \\
				 & (1.576) & (1.471) & (1.405) \\
				 Fraction County Hispanic & 0.107 & 0.037 & 0.143 \\
				 & (0.160) & (0.079) & (0.179) \\
				 \# School Teachers & 39.591 & 32.684 & 43.169 \\
				 & (30.764) & (24.090) & (33.146)\\
				 \# Hispanic Students & 164.343 & 68.500  & 214.011 \\
				 & (259.096) & (117.433) & (295.883) \\
				 \# Total Students & 581.524 & 478.166 & 635.086 \\
				 & (482.595) & (383.924) & (518.467) \\
				 Observations & 41,502 & 11,252 & 30,250 \\
				\hline\addlinespace
				\hline\addlinespace
				Panel C: Campaign Contributions & \multicolumn{3}{c}{} \\
				\hline\addlinespace
				Hispanic Trump Donations & 0.080 & 0.032 & 0.175 \\
				& (1.165) & (0.047) & (1.900) \\
				Hispanic Clinton Donations & 0.049 & 1.407 & 1.187 \\
				& (3.014) & (1.476) & (4.773)\\
				Log Income & 9.279 & 9.253 & 9.329 \\
				& (0.270) & (0.232) & (0.327) \\
				Log Population & 10.830 & 10.084 & 10.969 \\
				& (1.514) & (1.372) & (1.607) \\
				Fraction County Hispanic & 0.148 & 0.134 & 0.176 \\
				& (0.214) & (0.200) & (0.236) \\
				Observations & 619,011 & 411,673 & 207,338 \\
				\hline\addlinespace
				 \hline\addlinespace
			\end{tabular}
			\begin{tablenotes}[flushleft]
				\item \textit{Notes:} The table presents means (and standard deviations). Variables in Panel A and C aggregate data from firms and campaign contributions into 2 KM$^2$ grid points in Florida and the USA respectively. Variables in Panel B refer to our schools sample.  Column 1 shows data for all observations.  Columns 2 and 3 show data for the subsample without and with SLTV coverage, respectively. All panels only keep observations within 100 KM of the coverage contour. No control is significantly different across the coverage contour at the $\alpha = .1$ level.
			\end{tablenotes}
		\end{threeparttable}
	}
\end{table}

\begin{table}[!h]
	\centering
	\captionsetup{skip=1.5pt}
	\caption{Influence of Spanish Language Television on Migration Between Counties - Origin Sample} \label{t:mig_orig}
	\scalebox{.7}{
		\begin{threeparttable}
			\begin{tabular}{lcccccccccc}
				\hline\hline\addlinespace
				& \multicolumn{3}{c}{IHS(\# Hispanic Migrants)} \\
				\cline{2-4} 
				Panel A: Origin County Inside Contour&  (1) & (2) & (3) \\
                                \hline\addlinespace
Dummy: Destination Outside TV Contour & $-$0.387$^{***}$ & $-$0.286$^{***}$ & $-$0.280$^{***}$ \\ 
  & (0.048) & (0.044) & (0.044) \\ 
 TV Dummy $\times$ Distance to Origin & $-$0.003$^{**}$ & $-$0.004$^{***}$ & $-$0.004$^{***}$ \\ 
  & (0.001) & (0.001) & (0.001) \\ 
 TV Dummy $\times$ Distance to Destination & 0.001 & $-$0.002$^{*}$ & $-$0.002 \\ 
  & (0.001) & (0.001) & (0.001) \\ 
 Distance from Contour to Origin (KM) & 0.001 & 0.003$^{*}$ & 0.003 \\ 
  & (0.002) & (0.002) & (0.002) \\ 
 Distance from Contour to Destination (KM) & $-$0.001 & 0.002 & 0.002 \\ 
  & (0.001) & (0.001) & (0.001) \\ 
 Origin Log(Population) & 0.146$^{***}$ & 0.161$^{***}$ & 0.150$^{***}$ \\ 
  & (0.020) & (0.017) & (0.021) \\ 
 Destination Log(Population) & 0.150$^{***}$ & 0.136$^{***}$ & 0.125$^{***}$ \\ 
  & (0.014) & (0.013) & (0.016) \\ 
 Origin \% Hispanic &  & 0.792$^{***}$ & 0.881$^{***}$ \\ 
  &  & (0.103) & (0.141) \\ 
 Destination \% Hispanic &  & 1.485$^{***}$ & 1.573$^{***}$ \\ 
  &  & (0.122) & (0.141) \\ 
 Origin Log(Income) &  &  & 0.093 \\ 
  &  &  & (0.094) \\ 
 Destination Log(Income) &  &  & 0.090 \\ 
  &  &  & (0.078) \\ 
Observations & 8,479 & 8,479 & 8,479 \\ 
\hline\addlinespace
Panel B: Origin County Outside Contour & & & \\ 
\hline\addlinespace
 Dummy: Destination Inside TV Contour & $-$0.078 & $-$0.123 & $-$0.120 \\ 
  & (0.108) & (0.096) & (0.096) \\ 
 TV Dummy $\times$ Distance to Origin & $-$0.003$^{*}$ & $-$0.004$^{***}$ & $-$0.004$^{***}$ \\ 
  & (0.002) & (0.001) & (0.001) \\ 
 TV Dummy $\times$ Distance to Destination & $-$0.004$^{***}$ & $-$0.002 & $-$0.002 \\ 
  & (0.001) & (0.001) & (0.001) \\ 
 Distance from Contour to Origin (KM) & $-$0.0003 & 0.001 & 0.001 \\ 
  & (0.001) & (0.001) & (0.001) \\ 
 Distance from Contour to Destination (KM) & $-$0.001$^{***}$ & $-$0.001$^{***}$ & $-$0.001$^{***}$ \\ 
  & (0.0002) & (0.0003) & (0.0003) \\ 
 Origin Log(Population) & 0.164$^{***}$ & 0.131$^{***}$ & 0.094$^{***}$ \\ 
  & (0.017) & (0.021) & (0.026) \\ 
 Destination Log(Population) & 0.150$^{***}$ & 0.128$^{***}$ & 0.125$^{***}$ \\ 
  & (0.023) & (0.020) & (0.021) \\ 
 Origin \% Hispanic &  & 1.328$^{***}$ & 1.611$^{***}$ \\ 
  &  & (0.295) & (0.329) \\ 
 Destination \% Hispanic &  & 1.485$^{***}$ & 1.481$^{***}$ \\ 
  &  & (0.293) & (0.318) \\ 
 Origin Log(Income) &  &  & 0.407$^{**}$ \\ 
  &  &  & (0.193) \\ 
 Destination Log(Income) &  &  & 0.003 \\ 
  &  &  & (0.087) \\ 
Observations & 4,062 & 4,062 & 4,062 \\         
\hline\addlinespace
                                Origin F.E. & Yes & Yes  & Yes\\
				\addlinespace\hline\hline
			\end{tabular}
			\begin{tablenotes}[flushleft]
				\item \textit{Notes:} The table presents coefficient estimates from regressions at the county-county level, only keeping origin counties within 100 KM of a contour boundary. The dependent variables are inverse hyperbolic sine transformed counts of Hispanic migrants from the origin county to the destination county. The key dependent variable of interest is the TV Dummy, which tracks whether the destination county is inside or outside the TV contour. This is interacted with the distance to the boundary for both the origin and destination county. County controls include log income, log population, and percentage county Hispanic for both origin and destination county. All regressions also contain origin county fixed effects. Standard errors are given in parentheses. *, **, and *** denote statistical significance at the 10\%, 5\%, and 1\% levels, respectively.
			\end{tablenotes}
		\end{threeparttable}
	}
\end{table}
\begin{table}[!h]
	\centering
	\captionsetup{skip=1.5pt}
	\caption{Influence of Spanish Language Television on Migration Between Counties - Destination Sample} \label{t:mig_dest}
	\scalebox{.7}{
		\begin{threeparttable}
			\begin{tabular}{lcccccccccc}
				\hline\hline\addlinespace
				& \multicolumn{3}{c}{IHS(\# Hispanic Migrants)} \\
				\cline{2-4} 
				Panel A: Destination County Inside Contour&  (1) & (2) & (3) \\
                                \hline\addlinespace
 Dummy: Origin Outside TV Contour & $-$0.410$^{***}$ & $-$0.356$^{***}$ & $-$0.349$^{***}$ \\ 
  & (0.088) & (0.082) & (0.081) \\ 
 TV Dummy $\times$ Distance to Destination & $-$0.007$^{***}$ & $-$0.008$^{***}$ & $-$0.008$^{***}$ \\ 
  & (0.003) & (0.003) & (0.003) \\ 
 TV Dummy $\times$ Distance to Origin & $-$0.002 & $-$0.004$^{**}$ & $-$0.004$^{*}$ \\ 
  & (0.002) & (0.002) & (0.002) \\ 
 Distance from Contor to Destination (KM) & 0.002 & 0.004$^{**}$ & 0.004$^{**}$ \\ 
  & (0.002) & (0.002) & (0.002) \\ 
 Distance from Contour to Origin (KM) & 0.001 & 0.004 & 0.003 \\ 
  & (0.002) & (0.002) & (0.002) \\ 
 Destination Log(Population) & 0.179$^{***}$ & 0.181$^{***}$ & 0.175$^{***}$ \\ 
  & (0.019) & (0.016) & (0.019) \\ 
 Origin Log(Population) & 0.115$^{***}$ & 0.117$^{***}$ & 0.102$^{***}$ \\ 
  & (0.018) & (0.017) & (0.020) \\ 
 Destination \% Hispanic &  & 1.384$^{***}$ & 1.428$^{***}$ \\ 
  &  & (0.183) & (0.205) \\ 
 Origin \% Hispanic &  & 0.813$^{***}$ & 0.949$^{***}$ \\ 
  &  & (0.182) & (0.203) \\ 
 Destination Log(Income) &  &  & 0.041 \\ 
  &  &  & (0.099) \\ 
 Origin Log(Income) &  &  & 0.138 \\ 
  &  &  & (0.109) \\ 
Observations & 4,338 & 4,338 & 4,338 \\ 
\hline\addlinespace
Panel B: Origin County Outside Contour & & & \\ 
\hline\addlinespace
 Dummy: Origin Inside TV Contour & $-$0.140 & $-$0.194 & $-$0.193 \\ 
  & (0.152) & (0.144) & (0.144) \\ 
 TV Dummy $\times$ Distance to Destination & $-$0.004$^{*}$ & $-$0.007$^{***}$ & $-$0.007$^{***}$ \\ 
  & (0.002) & (0.002) & (0.002) \\ 
 TV Dummy $\times$ Distance to Origin & $-$0.007$^{**}$ & $-$0.004 & $-$0.004 \\ 
  & (0.003) & (0.003) & (0.003) \\ 
 Distance from Contor to Destination (KM) & $-$0.0003 & 0.002 & 0.002 \\ 
  & (0.002) & (0.001) & (0.001) \\ 
 Distance from Contour to Origin (KM) & $-$0.001$^{***}$ & $-$0.002$^{***}$ & $-$0.002$^{***}$ \\ 
  & (0.0004) & (0.0004) & (0.0004) \\ 
 Destination Log(Population) & 0.253$^{***}$ & 0.169$^{***}$ & 0.153$^{***}$ \\ 
  & (0.041) & (0.023) & (0.030) \\ 
 Origin Log(Population) & 0.182$^{***}$ & 0.181$^{***}$ & 0.181$^{***}$ \\ 
  & (0.035) & (0.030) & (0.034) \\ 
 Destination \% Hispanic &  & 2.324$^{***}$ & 2.471$^{***}$ \\ 
  &  & (0.389) & (0.411) \\ 
 Origin \% Hispanic &  & 1.276$^{**}$ & 1.253$^{**}$ \\ 
  &  & (0.602) & (0.584) \\ 
 Destination Log(Income) &  &  & 0.181 \\ 
  &  &  & (0.196) \\ 
 Origin Log(Income) &  &  & $-$0.015 \\ 
  &  &  & (0.192) \\ 
Observations & 1,659 & 1,659 & 1,659 \\        
\hline\addlinespace
                                Destination F.E. & Yes & Yes  & Yes\\
				\addlinespace\hline\hline
			\end{tabular}
			\begin{tablenotes}[flushleft]
				\item \textit{Notes:} The table presents coefficient estimates from regressions at the county-county level, only keeping destination counties within 100 KM of a contour boundary. The dependent variables are inverse hyperbolic sine transformed counts of Hispanic migrants from the origin county to the destination county. The key dependent variable of interest is the TV Dummy, which tracks whether the destination county is inside or outside the TV contour. This is interacted with the distance to the boundary for both the origin and destination county. County controls include log income, log population, and percentage county Hispanic for both origin and destination county. All regressions also contain destination county fixed effects. Standard errors are given in parentheses. *, **, and *** denote statistical significance at the 10\%, 5\%, and 1\% levels, respectively.
			\end{tablenotes}
		\end{threeparttable}
	}
\end{table}

%
% Table created by stargazer v.5.2.2 by Marek Hlavac, Harvard University. E-mail: hlavac at fas.harvard.edu
% Date and time: Mon, Feb 03, 2020 - 17:52:46
\begin{table}[!htbp] \centering 
  \caption{Effect of TV on Hispanic Donations to Trump, 100 KM Radius} 
  \label{} 
\begin{tabular}{@{\extracolsep{-5pt}}lcccc} 
\\[-1.8ex]\hline 
\hline \\[-1.8ex] 
 & \multicolumn{4}{c}{\textit{Dependent variable:}} \\ 
\cline{2-5} 
\\[-1.8ex] & \multicolumn{4}{c}{donations\_d} \\ 
\\[-1.8ex] & (1) & (2) & (3) & (4)\\ 
\hline \\[-1.8ex] 
 intersects & 37.502$^{***}$ & 20.178$^{**}$ & 22.202$^{***}$ & 18.342$^{**}$ \\ 
  & (9.181) & (8.286) & (8.402) & (8.016) \\ 
  & & & & \\ 
 distance & 0.701 & 1.169 & 1.166 & 1.235 \\ 
  & (1.047) & (0.943) & (0.943) & (0.899) \\ 
  & & & & \\ 
 dist2 & $-$0.006 & $-$0.008 & $-$0.009 & $-$0.008 \\ 
  & (0.010) & (0.009) & (0.009) & (0.009) \\ 
  & & & & \\ 
 logPop &  & 128.543$^{***}$ & 127.408$^{***}$ & 80.284$^{***}$ \\ 
  &  & (4.503) & (4.570) & (5.041) \\ 
  & & & & \\ 
 pcHispanic &  &  & $-$44.793 & 281.147$^{***}$ \\ 
  &  &  & (30.888) & (34.276) \\ 
  & & & & \\ 
 income &  &  &  & 0.051$^{***}$ \\ 
  &  &  &  & (0.003) \\ 
  & & & & \\ 
 intersects:distance & $-$0.263 & $-$1.155$^{*}$ & $-$1.201$^{*}$ & $-$1.440$^{**}$ \\ 
  & (0.710) & (0.640) & (0.641) & (0.611) \\ 
  & & & & \\ 
 intersects:dist2 & 0.063$^{***}$ & 0.057$^{***}$ & 0.057$^{***}$ & 0.060$^{***}$ \\ 
  & (0.010) & (0.009) & (0.009) & (0.009) \\ 
  & & & & \\ 
 Constant & 59.673$^{***}$ & $-$1,261.867$^{***}$ & $-$1,243.851$^{***}$ & $-$1,362.724$^{***}$ \\ 
  & (21.409) & (50.144) & (51.652) & (49.675) \\ 
  & & & & \\ 
\hline \\[-1.8ex] 
Observations & 3,479 & 3,479 & 3,479 & 3,479 \\ 
R$^{2}$ & 0.128 & 0.294 & 0.294 & 0.358 \\ 
Adjusted R$^{2}$ & 0.127 & 0.293 & 0.293 & 0.357 \\ 
\hline 
\hline \\[-1.8ex] 
\textit{Note:}  & \multicolumn{4}{r}{$^{*}$p$<$0.1; $^{**}$p$<$0.05; $^{***}$p$<$0.01} \\ 
\end{tabular} 
\end{table} 

%
% Table created by stargazer v.5.2.2 by Marek Hlavac, Harvard University. E-mail: hlavac at fas.harvard.edu
% Date and time: Mon, Feb 03, 2020 - 16:47:20
\begin{table}[!htbp] \centering 
  \caption{Effect of TV on Hispanic Donations to Clinton, 100 KM Radius} 
  \label{} 
\begin{tabular}{@{\extracolsep{-5pt}}lcccc} 
\\[-1.8ex]\hline 
\hline \\[-1.8ex] 
 & \multicolumn{4}{c}{\textit{Dependent variable:}} \\ 
\cline{2-5} 
\\[-1.8ex] & \multicolumn{4}{c}{donations} \\ 
\\[-1.8ex] & (1) & (2) & (3) & (4)\\ 
\hline \\[-1.8ex] 
 intersects & 4.863$^{*}$ & 2.698 & 0.132 & $-$0.844 \\ 
  & (2.558) & (2.489) & (2.513) & (2.481) \\ 
  & & & & \\ 
 distance & 0.0002 & 0.0001 & 0.00002 & 0.00000 \\ 
  & (0.0003) & (0.0003) & (0.0003) & (0.0003) \\ 
  & & & & \\ 
 dist2 & $-$0.000 & $-$0.000 & 0.000 & 0.000 \\ 
  & (0.000) & (0.000) & (0.000) & (0.000) \\ 
  & & & & \\ 
 logPop &  & 12.347$^{***}$ & 11.427$^{***}$ & 5.977$^{***}$ \\ 
  &  & (1.394) & (1.391) & (1.645) \\ 
  & & & & \\ 
 pcHispanic &  &  & 78.466$^{***}$ & 119.350$^{***}$ \\ 
  &  &  & (15.348) & (16.590) \\ 
  & & & & \\ 
 income &  &  &  & 0.004$^{***}$ \\ 
  &  &  &  & (0.001) \\ 
  & & & & \\ 
 intersects:distance & $-$0.0001 & $-$0.0001 & $-$0.0001 & $-$0.0001 \\ 
  & (0.0002) & (0.0002) & (0.0001) & (0.0001) \\ 
  & & & & \\ 
 intersects:dist2 & 0.000$^{*}$ & 0.000 & 0.000$^{*}$ & 0.000$^{*}$ \\ 
  & (0.000) & (0.000) & (0.000) & (0.000) \\ 
  & & & & \\ 
 Constant & 3.994 & $-$136.034$^{***}$ & $-$126.548$^{***}$ & $-$122.380$^{***}$ \\ 
  & (6.612) & (17.060) & (16.979) & (16.743) \\ 
  & & & & \\ 
\hline \\[-1.8ex] 
Observations & 1,164 & 1,164 & 1,164 & 1,164 \\ 
R$^{2}$ & 0.032 & 0.093 & 0.113 & 0.140 \\ 
Adjusted R$^{2}$ & 0.028 & 0.088 & 0.108 & 0.134 \\ 
\hline 
\hline \\[-1.8ex] 
\textit{Note:}  & \multicolumn{4}{r}{$^{*}$p$<$0.1; $^{**}$p$<$0.05; $^{***}$p$<$0.01} \\ 
\end{tabular} 
\end{table} 

%
% Table created by stargazer v.5.2.2 by Marek Hlavac, Harvard University. E-mail: hlavac at fas.harvard.edu
% Date and time: Mon, Feb 03, 2020 - 16:47:33
\begin{table}[!htbp] \centering 
  \caption{Effect of TV on Hispanic Donations to Clinton, 100 KM Radius} 
  \label{} 
\begin{tabular}{@{\extracolsep{-5pt}}lcccc} 
\\[-1.8ex]\hline 
\hline \\[-1.8ex] 
 & \multicolumn{4}{c}{\textit{Dependent variable:}} \\ 
\cline{2-5} 
\\[-1.8ex] & \multicolumn{4}{c}{donations\_d} \\ 
\\[-1.8ex] & (1) & (2) & (3) & (4)\\ 
\hline \\[-1.8ex] 
 intersects & 1.123$^{*}$ & 0.567 & $-$0.107 & $-$0.338 \\ 
  & (0.604) & (0.584) & (0.588) & (0.580) \\ 
  & & & & \\ 
 distance & 0.00002 & $-$0.00000 & $-$0.00003 & $-$0.00003 \\ 
  & (0.0001) & (0.0001) & (0.0001) & (0.0001) \\ 
  & & & & \\ 
 dist2 & $-$0.000 & 0.000 & 0.000 & 0.000 \\ 
  & (0.000) & (0.000) & (0.000) & (0.000) \\ 
  & & & & \\ 
 logPop &  & 3.174$^{***}$ & 2.933$^{***}$ & 1.641$^{***}$ \\ 
  &  & (0.327) & (0.325) & (0.385) \\ 
  & & & & \\ 
 pcHispanic &  &  & 20.607$^{***}$ & 30.295$^{***}$ \\ 
  &  &  & (3.589) & (3.878) \\ 
  & & & & \\ 
 income &  &  &  & 0.001$^{***}$ \\ 
  &  &  &  & (0.0002) \\ 
  & & & & \\ 
 intersects:distance & $-$0.00000 & $-$0.00000 & $-$0.00001 & $-$0.00001 \\ 
  & (0.00004) & (0.00004) & (0.00004) & (0.00003) \\ 
  & & & & \\ 
 intersects:dist2 & 0.000 & 0.000 & 0.000 & 0.000 \\ 
  & (0.000) & (0.000) & (0.000) & (0.000) \\ 
  & & & & \\ 
 Constant & 1.071 & $-$34.928$^{***}$ & $-$32.437$^{***}$ & $-$31.449$^{***}$ \\ 
  & (1.561) & (4.001) & (3.971) & (3.914) \\ 
  & & & & \\ 
\hline \\[-1.8ex] 
Observations & 1,164 & 1,164 & 1,164 & 1,164 \\ 
R$^{2}$ & 0.035 & 0.108 & 0.133 & 0.159 \\ 
Adjusted R$^{2}$ & 0.031 & 0.103 & 0.127 & 0.154 \\ 
\hline 
\hline \\[-1.8ex] 
\textit{Note:}  & \multicolumn{4}{r}{$^{*}$p$<$0.1; $^{**}$p$<$0.05; $^{***}$p$<$0.01} \\ 
\end{tabular} 
\end{table} 


%\begin{table}[!h]
	\centering
	\captionsetup{skip=1.5pt}
	\caption{Robustness of Influence of Spanish Language Television on Hispanic Business Ownership} \label{firm_busn}
	\scalebox{.8}{
		\begin{threeparttable}
			\begin{tabular}{lcccccccccc}
				\hline\hline\addlinespace
				 & \multicolumn{4}{c}{\textit{IHS(\# Hispanic Owned Businesses)}}\\
				&  (1) & (2) & (3) & (4) \\
                                \hline\addlinespace
 TV Dummy & 0.261$^{***}$ & 0.122$^{***}$ & 0.112$^{***}$ & 0.132$^{***}$ \\ 
  & (0.014) & (0.014) & (0.014) & (0.015) \\ 
 TV Dummy $\times$ Distance to Boundary & 0.010$^{***}$ & 0.007$^{***}$ & 0.007$^{***}$ & 0.007$^{***}$ \\ 
  & (0.001) & (0.001) & (0.001) & (0.001) \\ 
 Distance to Boundary (meters) & 0.006$^{***}$ & 0.009$^{***}$ & 0.010$^{***}$ & 0.011$^{***}$ \\ 
  & (0.001) & (0.001) & (0.001) & (0.001) \\ 
 Log(Population) &  & 0.412$^{***}$ & 0.388$^{***}$ & 0.342$^{***}$ \\ 
  &  & (0.011) & (0.012) & (0.014) \\ 
 County \% Hispanic &  &  & 1.261$^{***}$ & 1.414$^{***}$ \\ 
  &  &  & (0.133) & (0.136) \\ 
 Log(Income) &  &  &  & 0.391$^{***}$ \\ 
  &  &  &  & (0.070) \\ 
Observations & 23,853 & 23,853 & 23,853 & 23,853 \\ 
				\addlinespace\hline\hline
			\end{tabular}
			\begin{tablenotes}[flushleft]
				\item \textit{Notes:} The table presents coefficient estimates from regressions at aggregated into grids of size approximately 4 KM$^2$, only keeping grid points within 100 KM of a contour boundary. The dependent variable is the inverse hyperbolic sine transformed counts of Hispanic owned firms within the grid. The key dependent variable of interest is the TV Dummy, which tracks whether the school is within a coverage contour boundary for a Spanish language television station. This is interacted with the distance to the boundary. Controls are at the county level Standard errors are given in parentheses. *, **, and *** denote statistical significance at the 10\%, 5\%, and 1\% levels, respectively.
			\end{tablenotes}
		\end{threeparttable}
	}
\end{table}
%\begin{table}[!h]
	\centering
	\captionsetup{skip=1.5pt}
	\caption{Influence of Spanish Language Television on Businesses with Hispanic Names} \label{firm_name}
	\scalebox{.8}{
		\begin{threeparttable}
			\begin{tabular}{lcccccccccc}
				\hline\hline\addlinespace
				 & \multicolumn{6}{c}{\textit{Hispanic Named Business Dummy}}\\
				&  (1) & (2) & (3) & (4) & (5) & (6) \\
                                \hline\addlinespace
 TV Dummy & 0.839$^{***}$ & 0.638$^{***}$ & 0.637$^{***}$ & 0.769$^{***}$ & 0.849$^{***}$ & 0.775$^{***}$ \\ 
  & (0.052) & (0.066) & (0.066) & (0.071) & (0.077) & (0.071) \\ 
 TV Dummy $\times$ Distance to Boundary & 0.008$^{***}$ & 0.002 & 0.002 & 0.0002 & $-$0.0002 & 0.0002 \\ 
  & (0.002) & (0.002) & (0.002) & (0.002) & (0.002) & (0.002) \\ 
 Distance to Boundary (meters) & 0.010$^{**}$ & 0.021$^{***}$ & 0.021$^{***}$ & 0.031$^{***}$ & 0.035$^{***}$ & 0.031$^{***}$ \\ 
  & (0.004) & (0.004) & (0.005) & (0.005) & (0.005) & (0.005) \\ 
 Log(Population) &  & 0.957$^{***}$ & 0.979$^{***}$ & 0.702$^{***}$ & 0.761$^{***}$ & 0.701$^{***}$ \\ 
  &  & (0.052) & (0.070) & (0.074) & (0.081) & (0.074) \\ 
 County \% Hispanic &  &  & $-$0.151 & 1.428$^{***}$ & 1.514$^{***}$ & 1.434$^{***}$ \\ 
  &  &  & (0.312) & (0.367) & (0.388) & (0.368) \\ 
 Log(Income) &  &  &  & 2.350$^{***}$ & 2.534$^{***}$ & 2.356$^{***}$ \\ 
  &  &  &  & (0.319) & (0.344) & (0.320) \\ 
Observations & 23,853 & 23,853 & 23,853 & 23,853 & 23,853 & 23,853 \\ 
				\hline\hline\addlinespace
				Only Hispanic Owners & No & No & No & No & Yes & No \\
				Only Non-Hispanic Owners & No & No & No & No & No & Yes \\
				\addlinespace\hline\hline
			\end{tabular}
			\begin{tablenotes}[flushleft]
				\item \textit{Notes:} The table presents coefficient estimates from logit regressions at aggregated into grids of size approximately 4 KM$^2$, only keeping grid points within 100 KM of a contour boundary. The dependent variable is a dummy for whether there is a firm with a Hispanic name within the grid. The key dependent variable of interest is the TV Dummy, which tracks whether the school is within a coverage contour boundary for a Spanish language television station. This is interacted with the distance to the boundary. Controls are at the county level. Standard errors are given in parentheses. *, **, and *** denote statistical significance at the 10\%, 5\%, and 1\% levels, respectively.
			\end{tablenotes}
		\end{threeparttable}
	}
\end{table}
%\begin{table}[!h]
	\centering
	\captionsetup{skip=1.5pt}
	\caption{Robustness of Influence of Spanish Language Television on Hispanic Owned Businesses with Hispanic Names} \label{firm_robust}
	\scalebox{.8}{
		\begin{threeparttable}
			\begin{tabular}{lcccccccccc}
				\hline\hline\addlinespace
				 & \multicolumn{8}{c}{\textit{Hispanic Owned and Named Business Dummy}}\\
				&  (1) & (2) & (3) & (4) & (5) & (6) & (7) & (8) \\
                                \hline\addlinespace
 TV Dummy & 0.849$^{***}$ & 1.071$^{***}$ & 0.305$^{***}$ & .8677$^{***}$ & 0.927$^{***}$ & 0.596$^{***}$ & 0.624$^{***}$ & 1.144$^{***}$ \\ 
  & (0.077) & (0.115) & (0.078) & (0.079) & (0.098) & (0.118) & (0.078) & (0.076) \\ 
 TV Dummy $\times$ Distance to Boundary  & $-$0.0002 & $-$0.008 & $-$0.003 & $-$0.001 & $-$0.002 & 0.042$^{***}$ & 0.001 & $-$0.001 \\ 
  & (0.002) & (0.007) & (0.002) & (0.002) & (0.004) & (0.010) & (0.002) & (0.002) \\ 
 Distance to Boundary (meters) & 0.035$^{***}$ & 0.123$^{***}$ & 0.013$^{***}$ & 0.036$^{***}$ & 0.049$^{***}$ & $-$0.097$^{***}$ & 0.026$^{***}$ & 0.042$^{***}$ \\ 
  & (0.005) & (0.021) & (0.005) & (0.005) & (0.012) & (0.035) & (0.005) & (0.006) \\ 
 Total Businesses &  &  & 0.023$^{***}$ &  &  &  &  &  \\ 
  &  &  & (0.001) &  &  &  &  &  \\ 
\hline \\[-1.8ex] 
Observations & 23,853 & 23,853 & 23,853 & 23,853 & 20,404 & 14,386 & 10,598 & 95,373 \\ 				\hline\hline\addlinespace
County Controls & Yes & Yes & Yes & Yes & Yes & Yes & Yes & Yes \\
Distance Cutoff (KM) & 100 & 100 & 100 & 100 & 50 &25 & 100 &100 \\
Grid Size (KM$^2$) & 4 & 4 & 4 & 4 &4 & 4 & 9 & 1\\ 
Distance$^2$ & No & Yes & No & No & No & No & No & No \\
No Food Names & No & No & No & Yes & No & No & No & No \\
				\addlinespace\hline\hline
			\end{tabular}
			\begin{tablenotes}[flushleft]
				\item \textit{Notes:} The table presents coefficient estimates from logit regressions at aggregated into grids, only keeping grid points within a certain cutoff of a contour boundary. The dependent variable is a dummy for whether there is a firm with a Hispanic name owned by a Hispanic person within the grid. Column (1) is the same specification as Table \ref{firm_name} Column (5). The key independent variable of interest is the TV Dummy, which tracks whether the school is within a coverage contour boundary for a Spanish language television station. This is interacted with the distance to the boundary. Controls are at the county level, and include log population, log income, and percentage of county that is Hispanic. Total Businesses is the total number of businesses in the grid, while No Food Names removes references to various Hispanic foods as part of the criterion for selection of Hispanic business names. Various distance cut-offs, grid sizes, as well as the interaction with distance squared are presented. Standard errors are given in parentheses. *, **, and *** denote statistical significance at the 10\%, 5\%, and 1\% levels, respectively.
			\end{tablenotes}
		\end{threeparttable}
	}
\end{table}
%\begin{table}[!h]
	\centering
	\captionsetup{skip=1.5pt}
	\caption{Spatial Robustness of Influence of Spanish Language Television on Hispanic Firm Ownership} \label{firm_spatial}
	\scalebox{.8}{
		\begin{threeparttable}
			\begin{tabular}{lcccccccccc}
				\hline\hline\addlinespace
				 & \multicolumn{3}{c}{\textit{IHS(\# Hispanic Owned Firms)}}\\
				&  (1) & (2) & (3) \\
                                \hline\addlinespace
 TV Dummy & 0.122$^{***}$ & 0.022$^{***}$ & 0.126$^{***}$ \\ 
  & (0.014) & (0.006) & (0.036) \\ 
\hline\hline\addlinespace
Observations & 23,853 & 23,853 & 23,853 \\ 
Log Likelihood &  & $-$38,404 & $-$38,440 \\ 
$\sigma^{2}$ &  & 1.168 & 1.170 \\ 
Akaike Inf. Crit. &  & 76,821 & 76,894 \\ 
Wald Test (df = 1) &  & 65,139$^{***}$ & 63,913$^{***}$ \\ 
LR Test (df = 1) &  & 24,759$^{***}$ & 24,687$^{***}$ \\ 
\hline \addlinespace
                                County Controls & Yes & Yes  & Yes \\
                                Model & OLS & SAR Lag & SAR Error \\
				\addlinespace\hline\hline
			\end{tabular}
			\begin{tablenotes}[flushleft]
				\item \textit{Notes:} The table presents coefficient estimates from regressions at aggregated into grids of size approximately 4 KM$^2$, only keeping grid points within 100 KM of a contour boundary. The dependent variable is the inverse hyperbolic sine transformed counts of Hispanic owned firms in the grid. The key independent variable of interest is the TV Dummy, which tracks whether the school is within a coverage contour boundary for a Spanish language television station. This is interacted with the distance to the boundary. County controls include log population. Additionally controlling for log income and percentage county Hispanic for the county which the grid is in yields similar coefficients, although standard errors cannot be estimated due to computational limitations. The SAR Lag model is a spatial autoregressive lag model and the SAR Error model is a spatial autoregressive error model, both with weight matrices based on 4 nearest neighbours. Standard errors are given in parentheses. *, **, and *** denote statistical significance at the 10\%, 5\%, and 1\% levels, respectively.
			\end{tablenotes}
		\end{threeparttable}
	}
\end{table}

%
% Table created by stargazer v.5.2.2 by Marek Hlavac, Harvard University. E-mail: hlavac at fas.harvard.edu
% Date and time: Thu, Jan 23, 2020 - 17:05:20
\begin{table}[!htbp] \centering 
  \caption{Effect of TV on IHS(Hispanic Out of School Suspension)} 
  \label{} 
\begin{tabular}{@{\extracolsep{-2pt}}lcccc} 
\\[-1.8ex]\hline 
\hline \\[-1.8ex] 
 & \multicolumn{4}{c}{\textit{Dependent variable:}} \\ 
\cline{2-5} 
\\[-1.8ex] & \multicolumn{4}{c}{IHS(\# Hispanic Out of School Suspension)} \\ 
\\[-1.8ex] & (1) & (2) & (3) & (4)\\ 
\hline \\[-1.8ex] 
 TV Dummy & 0.189$^{***}$ & 0.053$^{***}$ & 0.072$^{***}$ & 0.033$^{**}$ \\ 
  & (0.020) & (0.016) & (0.016) & (0.016) \\ 
  & & & & \\ 
 TV Dummy $\times$ Distance to Boundary & 0.013$^{***}$ & 0.003$^{***}$ & 0.005$^{***}$ & 0.005$^{***}$ \\ 
  & (0.001) & (0.001) & (0.001) & (0.001) \\ 
  & & & & \\ 
 TV Dummy $\times$ Distance2 & $-$0.0002$^{***}$ & $-$0.00001 & $-$0.00003 & $-$0.00002 \\ 
  & (0.00002) & (0.00002) & (0.00002) & (0.00002) \\ 
  & & & & \\ 
 Distance to Boundary (meters) & $-$0.006$^{***}$ & $-$0.004$^{***}$ & $-$0.004$^{***}$ & $-$0.006$^{***}$ \\ 
  & (0.001) & (0.001) & (0.001) & (0.001) \\ 
  & & & & \\ 
 Distance2 & 0.00005$^{***}$ & 0.00004$^{***}$ & 0.00004$^{***}$ & 0.00005$^{***}$ \\ 
  & (0.00001) & (0.00001) & (0.00001) & (0.00001) \\ 
  & & & & \\ 
 \% County Hispanic & 1.356$^{***}$ & $-$0.300$^{***}$ & $-$0.326$^{***}$ & $-$0.550$^{***}$ \\ 
  & (0.044) & (0.041) & (0.040) & (0.042) \\ 
  & & & & \\ 
 Log(Population) & $-$0.218$^{***}$ & $-$0.430$^{***}$ & $-$0.371$^{***}$ & $-$0.575$^{***}$ \\ 
  & (0.023) & (0.019) & (0.019) & (0.022) \\ 
  & & & & \\ 
 \# Teachers at School &  & 0.007$^{***}$ & 0.005$^{***}$ & 0.006$^{***}$ \\ 
  &  & (0.0003) & (0.0003) & (0.0003) \\ 
  & & & & \\ 
 \# Hispanic Students &  & 0.002$^{***}$ & 0.002$^{***}$ & 0.002$^{***}$ \\ 
  &  & (0.00003) & (0.00003) & (0.00003) \\ 
  & & & & \\ 
 Total Students &  & 0.0001$^{***}$ & 0.0001$^{***}$ & 0.00004$^{*}$ \\ 
  &  & (0.00002) & (0.00002) & (0.00002) \\ 
  & & & & \\ 
 Contains Grade 1 &  &  & $-$0.545$^{***}$ & $-$0.558$^{***}$ \\ 
  &  &  & (0.011) & (0.011) \\ 
  & & & & \\ 
 Contains Grade 6 &  &  & 0.202$^{***}$ & 0.192$^{***}$ \\ 
  &  &  & (0.010) & (0.010) \\ 
  & & & & \\ 
 Contains Grade 9 &  &  & 0.011 & 0.010 \\ 
  &  &  & (0.013) & (0.013) \\ 
  & & & & \\ 
 Log(Income) &  &  &  & 0.067$^{***}$ \\ 
  &  &  &  & (0.004) \\ 
  & & & & \\ 
\hline \\[-1.8ex] 
Observations & 45,947 & 45,947 & 45,947 & 45,947 \\ 
R$^{2}$ & 0.067 & 0.344 & 0.400 & 0.404 \\ 
Adjusted R$^{2}$ & 0.067 & 0.344 & 0.400 & 0.403 \\ 
\hline 
\hline \\[-1.8ex] 
\textit{Note:}  & \multicolumn{4}{r}{$^{*}$p$<$0.1; $^{**}$p$<$0.05; $^{***}$p$<$0.01} \\ 
\end{tabular} 
\end{table} 

%
% Table created by stargazer v.5.2.2 by Marek Hlavac, Harvard University. E-mail: hlavac at fas.harvard.edu
% Date and time: Sat, Feb 08, 2020 - 18:29:08
\begin{table}[!htbp] \centering 
  \caption{Effect of TV on IHS(Hispanic \# Harassment Victims)} 
  \label{} 
\begin{tabular}{@{\extracolsep{-2pt}}lcccc} 
\\[-1.8ex]\hline 
\hline \\[-1.8ex] 
 & \multicolumn{4}{c}{\textit{Dependent variable:}} \\ 
\cline{2-5} 
\\[-1.8ex] & \multicolumn{4}{c}{IHS(\# Hispanic Victims of Harassment)} \\ 
\\[-1.8ex] & (1) & (2) & (3) & (4)\\ 
\hline \\[-1.8ex] 
 TV Dummy & $-$0.0003 & $-$0.001 & $-$0.001 & $-$0.0005 \\ 
  & (0.002) & (0.002) & (0.002) & (0.002) \\ 
  & & & & \\ 
 TV Dummy $\times$ Distance to Boundary & 0.0001 & 0.0001 & 0.0001 & 0.0001 \\ 
  & (0.0001) & (0.0001) & (0.0001) & (0.0001) \\ 
  & & & & \\ 
 TV Dummy $\times$ Distance$^{2}$ & $-$0.00000$^{*}$ & $-$0.00000$^{**}$ & $-$0.00000$^{**}$ & $-$0.00000$^{**}$ \\ 
  & (0.00000) & (0.00000) & (0.00000) & (0.00000) \\ 
  & & & & \\ 
 Distance to Boundary (meters) & $-$0.001$^{***}$ & $-$0.001$^{***}$ & $-$0.001$^{***}$ & $-$0.001$^{***}$ \\ 
  & (0.0002) & (0.0002) & (0.0002) & (0.0002) \\ 
  & & & & \\ 
 Distance$^{2}$ & 0.00001$^{***}$ & 0.00001$^{***}$ & 0.00001$^{***}$ & 0.00001$^{***}$ \\ 
  & (0.00000) & (0.00000) & (0.00000) & (0.00000) \\ 
  & & & & \\ 
 \% County Hispanic & 0.028$^{**}$ & 0.006 & 0.005 & 0.016 \\ 
  & (0.012) & (0.013) & (0.013) & (0.013) \\ 
  & & & & \\ 
 Log(Population) & 0.066$^{***}$ & 0.051$^{***}$ & 0.055$^{***}$ & 0.069$^{***}$ \\ 
  & (0.005) & (0.005) & (0.005) & (0.006) \\ 
  & & & & \\ 
 \# Teachers at School &  & 0.001$^{***}$ & 0.001$^{***}$ & 0.001$^{***}$ \\ 
  &  & (0.0001) & (0.0001) & (0.0001) \\ 
  & & & & \\ 
 \# Hispanic Students &  & 0.00003$^{***}$ & 0.00003$^{***}$ & 0.00004$^{***}$ \\ 
  &  & (0.00001) & (0.00001) & (0.00001) \\ 
  & & & & \\ 
 Total Students &  & $-$0.00003$^{***}$ & $-$0.00003$^{***}$ & $-$0.00002$^{***}$ \\ 
  &  & (0.00001) & (0.00001) & (0.00001) \\ 
  & & & & \\ 
 Contains Grade 1 &  &  & $-$0.037$^{***}$ & $-$0.036$^{***}$ \\ 
  &  &  & (0.003) & (0.003) \\ 
  & & & & \\ 
 Contains Grade 6 &  &  & 0.028$^{***}$ & 0.029$^{***}$ \\ 
  &  &  & (0.003) & (0.003) \\ 
  & & & & \\ 
 Contains Grade 9 &  &  & $-$0.010$^{***}$ & $-$0.010$^{**}$ \\ 
  &  &  & (0.004) & (0.004) \\ 
  & & & & \\ 
 Log(Income) &  &  &  & $-$0.005$^{***}$ \\ 
  &  &  &  & (0.001) \\ 
  & & & & \\ 
\hline \\[-1.8ex] 
Observations & 40,811 & 40,811 & 40,811 & 40,811 \\ 
R$^{2}$ & 0.009 & 0.016 & 0.023 & 0.023 \\ 
Adjusted R$^{2}$ & 0.009 & 0.016 & 0.023 & 0.023 \\ 
\hline 
\hline \\[-1.8ex] 
\textit{Note:}  & \multicolumn{4}{r}{$^{*}$p$<$0.1; $^{**}$p$<$0.05; $^{***}$p$<$0.01} \\ 
\end{tabular} 
\end{table} 

%
% Table created by stargazer v.5.2.2 by Marek Hlavac, Harvard University. E-mail: hlavac at fas.harvard.edu
% Date and time: Sun, Feb 02, 2020 - 15:48:38
\begin{table}[!htbp] \centering 
  \caption{Effect of TV on IHS(APs Taken)} 
  \label{} 
\begin{tabular}{@{\extracolsep{-2pt}}lcccc} 
\\[-1.8ex]\hline 
\hline \\[-1.8ex] 
 & \multicolumn{4}{c}{\textit{Dependent variable:}} \\ 
\cline{2-5} 
\\[-1.8ex] & \multicolumn{4}{c}{IHS(APs Taken by Hispanic Students)} \\ 
\\[-1.8ex] & (1) & (2) & (3) & (4)\\ 
\hline \\[-1.8ex] 
 TV Dummy & 0.307$^{***}$ & 0.223$^{***}$ & 0.232$^{***}$ & 0.166$^{***}$ \\ 
  & (0.065) & (0.048) & (0.047) & (0.047) \\ 
  & & & & \\ 
 TV Dummy $\times$ Distance to Boundary & 0.016$^{***}$ & 0.007$^{*}$ & 0.006$^{*}$ & 0.008$^{**}$ \\ 
  & (0.005) & (0.004) & (0.004) & (0.004) \\ 
  & & & & \\ 
 TV Dummy $\times$ Distance2 & $-$0.0001$^{*}$ & $-$0.00002 & $-$0.00002 & $-$0.00002 \\ 
  & (0.0001) & (0.0001) & (0.0001) & (0.0001) \\ 
  & & & & \\ 
 Distance to Boundary (meters) & $-$0.0002 & 0.003 & 0.003 & $-$0.002 \\ 
  & (0.004) & (0.003) & (0.003) & (0.003) \\ 
  & & & & \\ 
 Distance2 & $-$0.00005 & $-$0.0001$^{*}$ & $-$0.0001$^{**}$ & $-$0.00002 \\ 
  & (0.00005) & (0.00003) & (0.00003) & (0.00003) \\ 
  & & & & \\ 
 \% County Hispanic & 2.358$^{***}$ & 1.012$^{***}$ & 1.042$^{***}$ & 0.764$^{***}$ \\ 
  & (0.124) & (0.108) & (0.107) & (0.111) \\ 
  & & & & \\ 
 Log(Population) & $-$0.319$^{***}$ & $-$0.033 & $-$0.044 & $-$0.266$^{***}$ \\ 
  & (0.072) & (0.054) & (0.054) & (0.060) \\ 
  & & & & \\ 
 \# Teachers at School &  & $-$0.005$^{***}$ & $-$0.005$^{***}$ & $-$0.005$^{***}$ \\ 
  &  & (0.0005) & (0.0005) & (0.0005) \\ 
  & & & & \\ 
 \# Hispanic Students &  & 0.001$^{***}$ & 0.001$^{***}$ & 0.001$^{***}$ \\ 
  &  & (0.00003) & (0.00003) & (0.00003) \\ 
  & & & & \\ 
 Total Students &  & 0.0003$^{***}$ & 0.0003$^{***}$ & 0.0003$^{***}$ \\ 
  &  & (0.00003) & (0.00003) & (0.00003) \\ 
  & & & & \\ 
 Contains Grade 1 &  &  & $-$0.532$^{***}$ & $-$0.564$^{***}$ \\ 
  &  &  & (0.126) & (0.124) \\ 
  & & & & \\ 
 Contains Grade 6 &  &  & $-$0.170$^{**}$ & $-$0.225$^{***}$ \\ 
  &  &  & (0.068) & (0.067) \\ 
  & & & & \\ 
 Contains Grade 9 &  &  & 0.153$^{*}$ & 0.189$^{**}$ \\ 
  &  &  & (0.079) & (0.078) \\ 
  & & & & \\ 
 Log(Income) &  &  &  & 0.098$^{***}$ \\ 
  &  &  &  & (0.012) \\ 
  & & & & \\ 
\hline \\[-1.8ex] 
Observations & 2,342 & 2,342 & 2,342 & 2,342 \\ 
R$^{2}$ & 0.311 & 0.626 & 0.634 & 0.644 \\ 
Adjusted R$^{2}$ & 0.309 & 0.624 & 0.632 & 0.642 \\ 
\hline 
\hline \\[-1.8ex] 
\textit{Note:}  & \multicolumn{4}{r}{$^{*}$p$<$0.1; $^{**}$p$<$0.05; $^{***}$p$<$0.01} \\ 
\end{tabular} 
\end{table} 

%
% Table created by stargazer v.5.2.2 by Marek Hlavac, Harvard University. E-mail: hlavac at fas.harvard.edu
% Date and time: Sun, Feb 02, 2020 - 15:48:41
\begin{table}[!htbp] \centering 
  \caption{Effect of TV on IHS(APs Passed)} 
  \label{} 
\begin{tabular}{@{\extracolsep{-2pt}}lcccc} 
\\[-1.8ex]\hline 
\hline \\[-1.8ex] 
 & \multicolumn{4}{c}{\textit{Dependent variable:}} \\ 
\cline{2-5} 
\\[-1.8ex] & \multicolumn{4}{c}{IHS(APs Passed by Hispanic Students)} \\ 
\\[-1.8ex] & (1) & (2) & (3) & (4)\\ 
\hline \\[-1.8ex] 
 TV Dummy & 0.305$^{***}$ & 0.242$^{***}$ & 0.251$^{***}$ & 0.184$^{***}$ \\ 
  & (0.061) & (0.052) & (0.052) & (0.052) \\ 
  & & & & \\ 
 TV Dummy $\times$ Distance to Boundary & 0.005 & $-$0.003 & $-$0.004 & $-$0.002 \\ 
  & (0.005) & (0.004) & (0.004) & (0.004) \\ 
  & & & & \\ 
 TV Dummy $\times$ Distance2 & $-$0.00004 & 0.00005 & 0.0001 & 0.00005 \\ 
  & (0.0001) & (0.0001) & (0.0001) & (0.0001) \\ 
  & & & & \\ 
 Distance to Boundary (meters) & 0.005 & 0.007$^{**}$ & 0.008$^{**}$ & 0.003 \\ 
  & (0.004) & (0.003) & (0.003) & (0.003) \\ 
  & & & & \\ 
 Distance2 & $-$0.0001$^{*}$ & $-$0.0001$^{***}$ & $-$0.0001$^{***}$ & $-$0.0001 \\ 
  & (0.00004) & (0.00004) & (0.00004) & (0.00004) \\ 
  & & & & \\ 
 \% County Hispanic & 1.902$^{***}$ & 1.306$^{***}$ & 1.332$^{***}$ & 1.053$^{***}$ \\ 
  & (0.118) & (0.117) & (0.117) & (0.122) \\ 
  & & & & \\ 
 Log(Population) & 0.144$^{**}$ & 0.383$^{***}$ & 0.377$^{***}$ & 0.153$^{**}$ \\ 
  & (0.069) & (0.058) & (0.059) & (0.065) \\ 
  & & & & \\ 
 \# Teachers at School &  & $-$0.005$^{***}$ & $-$0.005$^{***}$ & $-$0.004$^{***}$ \\ 
  &  & (0.001) & (0.001) & (0.001) \\ 
  & & & & \\ 
 \# Hispanic Students &  & 0.001$^{***}$ & 0.001$^{***}$ & 0.001$^{***}$ \\ 
  &  & (0.00004) & (0.00004) & (0.00004) \\ 
  & & & & \\ 
 Total Students &  & 0.0004$^{***}$ & 0.0004$^{***}$ & 0.0004$^{***}$ \\ 
  &  & (0.00003) & (0.00003) & (0.00003) \\ 
  & & & & \\ 
 Contains Grade 1 &  &  & $-$0.216 & $-$0.248$^{*}$ \\ 
  &  &  & (0.137) & (0.136) \\ 
  & & & & \\ 
 Contains Grade 6 &  &  & $-$0.186$^{**}$ & $-$0.241$^{***}$ \\ 
  &  &  & (0.074) & (0.074) \\ 
  & & & & \\ 
 Contains Grade 9 &  &  & 0.133 & 0.169$^{**}$ \\ 
  &  &  & (0.086) & (0.085) \\ 
  & & & & \\ 
 Log(Income) &  &  &  & 0.098$^{***}$ \\ 
  &  &  &  & (0.013) \\ 
  & & & & \\ 
\hline \\[-1.8ex] 
Observations & 2,342 & 2,342 & 2,342 & 2,342 \\ 
R$^{2}$ & 0.195 & 0.429 & 0.433 & 0.447 \\ 
Adjusted R$^{2}$ & 0.193 & 0.426 & 0.430 & 0.443 \\ 
\hline 
\hline \\[-1.8ex] 
\textit{Note:}  & \multicolumn{4}{r}{$^{*}$p$<$0.1; $^{**}$p$<$0.05; $^{***}$p$<$0.01} \\ 
\end{tabular} 
\end{table} 

%
% Table created by stargazer v.5.2.2 by Marek Hlavac, Harvard University. E-mail: hlavac at fas.harvard.edu
% Date and time: Sat, Feb 08, 2020 - 18:23:06
\begin{table}[!htbp] \centering 
  \caption{Effect of TV on IHS(LEP)} 
  \label{} 
\begin{tabular}{@{\extracolsep{-2pt}}lcccc} 
\\[-1.8ex]\hline 
\hline \\[-1.8ex] 
 & \multicolumn{4}{c}{\textit{Dependent variable:}} \\ 
\cline{2-5} 
\\[-1.8ex] & \multicolumn{4}{c}{IHS(Hispanic \# Limited English Proficiency)} \\ 
\\[-1.8ex] & (1) & (2) & (3) & (4)\\ 
\hline \\[-1.8ex] 
 TV Dummy & 0.388$^{***}$ & 0.123$^{***}$ & 0.079$^{***}$ & 0.068$^{***}$ \\ 
  & (0.027) & (0.023) & (0.022) & (0.022) \\ 
  & & & & \\ 
 TV Dummy $\times$ Distance to Boundary & 0.013$^{***}$ & 0.010$^{***}$ & 0.009$^{***}$ & 0.009$^{***}$ \\ 
  & (0.001) & (0.001) & (0.001) & (0.001) \\ 
  & & & & \\ 
 Distance to Boundary (meters) & $-$0.006$^{***}$ & $-$0.005$^{***}$ & $-$0.004$^{***}$ & $-$0.005$^{***}$ \\ 
  & (0.0004) & (0.0003) & (0.0003) & (0.0003) \\ 
  & & & & \\ 
 \% County Hispanic & 4.237$^{***}$ & 0.977$^{***}$ & 1.061$^{***}$ & 0.994$^{***}$ \\ 
  & (0.066) & (0.062) & (0.060) & (0.063) \\ 
  & & & & \\ 
 Log(Population) & 0.561$^{***}$ & 0.367$^{***}$ & 0.253$^{***}$ & 0.191$^{***}$ \\ 
  & (0.035) & (0.029) & (0.028) & (0.033) \\ 
  & & & & \\ 
 \# Teachers at School &  & $-$0.0001 & 0.002$^{***}$ & 0.003$^{***}$ \\ 
  &  & (0.001) & (0.0005) & (0.0005) \\ 
  & & & & \\ 
 \# Hispanic Students &  & 0.005$^{***}$ & 0.004$^{***}$ & 0.004$^{***}$ \\ 
  &  & (0.00004) & (0.00004) & (0.00004) \\ 
  & & & & \\ 
 Total Students &  & 0.0001$^{***}$ & 0.0003$^{***}$ & 0.0003$^{***}$ \\ 
  &  & (0.00003) & (0.00003) & (0.00003) \\ 
  & & & & \\ 
 Contains Grade 1 &  &  & 0.338$^{***}$ & 0.334$^{***}$ \\ 
  &  &  & (0.016) & (0.016) \\ 
  & & & & \\ 
 Contains Grade 6 &  &  & $-$0.278$^{***}$ & $-$0.281$^{***}$ \\ 
  &  &  & (0.015) & (0.015) \\ 
  & & & & \\ 
 Contains Grade 9 &  &  & $-$0.840$^{***}$ & $-$0.840$^{***}$ \\ 
  &  &  & (0.019) & (0.019) \\ 
  & & & & \\ 
 Log(Income) &  &  &  & 0.020$^{***}$ \\ 
  &  &  &  & (0.006) \\ 
  & & & & \\ 
\hline \\[-1.8ex] 
Observations & 46,709 & 46,709 & 46,709 & 46,709 \\ 
R$^{2}$ & 0.175 & 0.427 & 0.479 & 0.479 \\ 
Adjusted R$^{2}$ & 0.175 & 0.427 & 0.479 & 0.479 \\ 
\hline 
\hline \\[-1.8ex] 
\textit{Note:}  & \multicolumn{4}{r}{$^{*}$p$<$0.1; $^{**}$p$<$0.05; $^{***}$p$<$0.01} \\ 
\end{tabular} 
\end{table} 
 % motivate with this, then add square
%
% Table created by stargazer v.5.2.2 by Marek Hlavac, Harvard University. E-mail: hlavac at fas.harvard.edu
% Date and time: Sun, Feb 02, 2020 - 15:54:08
\begin{table}[!htbp] \centering 
  \caption{Effect of TV on IHS(Gifted)} 
  \label{} 
\begin{tabular}{@{\extracolsep{-2pt}}lcccc} 
\\[-1.8ex]\hline 
\hline \\[-1.8ex] 
 & \multicolumn{4}{c}{\textit{Dependent variable:}} \\ 
\cline{2-5} 
\\[-1.8ex] & \multicolumn{4}{c}{IHS(Hispanic \# Gifted Students)} \\ 
\\[-1.8ex] & (1) & (2) & (3) & (4)\\ 
\hline \\[-1.8ex] 
 TV Dummy & 0.228$^{***}$ & 0.074$^{***}$ & 0.080$^{***}$ & 0.068$^{***}$ \\ 
  & (0.025) & (0.021) & (0.021) & (0.021) \\ 
  & & & & \\ 
 TV Dummy $\times$ Distance to Boundary & 0.029$^{***}$ & 0.022$^{***}$ & 0.022$^{***}$ & 0.022$^{***}$ \\ 
  & (0.002) & (0.002) & (0.002) & (0.002) \\ 
  & & & & \\ 
 TV Dummy $\times$ Distance2 & $-$0.0003$^{***}$ & $-$0.0002$^{***}$ & $-$0.0002$^{***}$ & $-$0.0002$^{***}$ \\ 
  & (0.00003) & (0.00002) & (0.00002) & (0.00002) \\ 
  & & & & \\ 
 Distance to Boundary (meters) & $-$0.009$^{***}$ & $-$0.008$^{***}$ & $-$0.008$^{***}$ & $-$0.009$^{***}$ \\ 
  & (0.001) & (0.001) & (0.001) & (0.001) \\ 
  & & & & \\ 
 Distance2 & 0.0001$^{***}$ & 0.0001$^{***}$ & 0.0001$^{***}$ & 0.0001$^{***}$ \\ 
  & (0.00001) & (0.00001) & (0.00001) & (0.00001) \\ 
  & & & & \\ 
 \% County Hispanic & 4.585$^{***}$ & 2.582$^{***}$ & 2.644$^{***}$ & 2.531$^{***}$ \\ 
  & (0.059) & (0.057) & (0.056) & (0.060) \\ 
  & & & & \\ 
 Log(Population) & 0.952$^{***}$ & 0.563$^{***}$ & 0.630$^{***}$ & 0.524$^{***}$ \\ 
  & (0.036) & (0.031) & (0.031) & (0.037) \\ 
  & & & & \\ 
 \# Teachers at School &  & 0.002$^{***}$ & 0.001 & 0.001 \\ 
  &  & (0.0005) & (0.0005) & (0.0005) \\ 
  & & & & \\ 
 \# Hispanic Students &  & 0.002$^{***}$ & 0.002$^{***}$ & 0.002$^{***}$ \\ 
  &  & (0.00004) & (0.00004) & (0.00004) \\ 
  & & & & \\ 
 Total Students &  & 0.001$^{***}$ & 0.001$^{***}$ & 0.001$^{***}$ \\ 
  &  & (0.00003) & (0.00003) & (0.00003) \\ 
  & & & & \\ 
 Contains Grade 1 &  &  & $-$0.441$^{***}$ & $-$0.445$^{***}$ \\ 
  &  &  & (0.017) & (0.017) \\ 
  & & & & \\ 
 Contains Grade 6 &  &  & 0.062$^{***}$ & 0.061$^{***}$ \\ 
  &  &  & (0.015) & (0.015) \\ 
  & & & & \\ 
 Contains Grade 9 &  &  & $-$0.297$^{***}$ & $-$0.292$^{***}$ \\ 
  &  &  & (0.021) & (0.021) \\ 
  & & & & \\ 
 Log(Income) &  &  &  & 0.030$^{***}$ \\ 
  &  &  &  & (0.006) \\ 
  & & & & \\ 
\hline \\[-1.8ex] 
Observations & 28,577 & 28,577 & 28,577 & 28,577 \\ 
R$^{2}$ & 0.309 & 0.516 & 0.532 & 0.533 \\ 
Adjusted R$^{2}$ & 0.309 & 0.516 & 0.532 & 0.532 \\ 
\hline 
\hline \\[-1.8ex] 
\textit{Note:}  & \multicolumn{4}{r}{$^{*}$p$<$0.1; $^{**}$p$<$0.05; $^{***}$p$<$0.01} \\ 
\end{tabular} 
\end{table} 

%
% Table created by stargazer v.5.2.2 by Marek Hlavac, Harvard University. E-mail: hlavac at fas.harvard.edu
% Date and time: Sun, Feb 02, 2020 - 15:54:11
\begin{table}[!htbp] \centering 
  \caption{Effect of TV on IHS(Gifted)} 
  \label{} 
\begin{tabular}{@{\extracolsep{-2pt}}lcccc} 
\\[-1.8ex]\hline 
\hline \\[-1.8ex] 
 & \multicolumn{4}{c}{\textit{Dependent variable:}} \\ 
\cline{2-5} 
\\[-1.8ex] & \multicolumn{4}{c}{IHS(Hispanic \# Gifted Students)} \\ 
\\[-1.8ex] & (1) & (2) & (3) & (4)\\ 
\hline \\[-1.8ex] 
 TV Dummy & 0.333$^{***}$ & 0.149$^{***}$ & 0.155$^{***}$ & 0.144$^{***}$ \\ 
  & (0.024) & (0.020) & (0.020) & (0.020) \\ 
  & & & & \\ 
 TV Dummy $\times$ Distance to Boundary & 0.009$^{***}$ & 0.008$^{***}$ & 0.008$^{***}$ & 0.008$^{***}$ \\ 
  & (0.001) & (0.001) & (0.001) & (0.001) \\ 
  & & & & \\ 
 Distance to Boundary (meters) & $-$0.003$^{***}$ & $-$0.003$^{***}$ & $-$0.003$^{***}$ & $-$0.003$^{***}$ \\ 
  & (0.0003) & (0.0003) & (0.0003) & (0.0003) \\ 
  & & & & \\ 
 \% County Hispanic & 4.584$^{***}$ & 2.578$^{***}$ & 2.640$^{***}$ & 2.530$^{***}$ \\ 
  & (0.059) & (0.057) & (0.056) & (0.060) \\ 
  & & & & \\ 
 Log(Population) & 0.960$^{***}$ & 0.565$^{***}$ & 0.630$^{***}$ & 0.527$^{***}$ \\ 
  & (0.036) & (0.031) & (0.031) & (0.037) \\ 
  & & & & \\ 
 \# Teachers at School &  & 0.002$^{***}$ & 0.001 & 0.001$^{*}$ \\ 
  &  & (0.0005) & (0.0005) & (0.0005) \\ 
  & & & & \\ 
 \# Hispanic Students &  & 0.002$^{***}$ & 0.002$^{***}$ & 0.002$^{***}$ \\ 
  &  & (0.00004) & (0.00004) & (0.00004) \\ 
  & & & & \\ 
 Total Students &  & 0.001$^{***}$ & 0.001$^{***}$ & 0.001$^{***}$ \\ 
  &  & (0.00003) & (0.00003) & (0.00003) \\ 
  & & & & \\ 
 Contains Grade 1 &  &  & $-$0.442$^{***}$ & $-$0.446$^{***}$ \\ 
  &  &  & (0.017) & (0.017) \\ 
  & & & & \\ 
 Contains Grade 6 &  &  & 0.059$^{***}$ & 0.058$^{***}$ \\ 
  &  &  & (0.015) & (0.015) \\ 
  & & & & \\ 
 Contains Grade 9 &  &  & $-$0.303$^{***}$ & $-$0.298$^{***}$ \\ 
  &  &  & (0.021) & (0.021) \\ 
  & & & & \\ 
 Log(Income) &  &  &  & 0.029$^{***}$ \\ 
  &  &  &  & (0.006) \\ 
  & & & & \\ 
\hline \\[-1.8ex] 
Observations & 28,577 & 28,577 & 28,577 & 28,577 \\ 
R$^{2}$ & 0.306 & 0.514 & 0.531 & 0.531 \\ 
Adjusted R$^{2}$ & 0.306 & 0.514 & 0.530 & 0.531 \\ 
\hline 
\hline \\[-1.8ex] 
\textit{Note:}  & \multicolumn{4}{r}{$^{*}$p$<$0.1; $^{**}$p$<$0.05; $^{***}$p$<$0.01} \\ 
\end{tabular} 
\end{table} 

%\begin{table}[!h]
	\centering
	\captionsetup{skip=1.5pt}
	\caption{Influence of Spanish Language Television on Hispanic Academic Achievement} \label{edu_top}
	\scalebox{.8}{
		\begin{threeparttable}
			\begin{tabular}{lcccccccccc}
				\hline\hline\addlinespace
				Panel A: IHS(\# Hispanic Gifted Students) &  (1) & (2) & (3) \\
                                \hline\addlinespace
 TV Dummy & 0.016$^{***}$ & 0.015$^{**}$ & 0.013$^{**}$ \\ 
  & (0.006) & (0.006) & (0.006) \\ 
 TV Dummy $\times$ Distance to Boundary & 0.001$^{***}$ & 0.001$^{***}$ & 0.001$^{***}$ \\ 
  & (0.0001) & (0.0001) & (0.0001) \\ 
 Distance to Boundary (meters) & 0.0002 & $-$0.0002 & $-$0.0002 \\ 
  & (0.0003) & (0.0003) & (0.0003) \\ 
 \# Hispanic Students & 0.003$^{***}$ & 0.002$^{***}$ & 0.002$^{***}$ \\ 
  & (0.00003) & (0.00004) & (0.00004) \\ 
Observations & 26,065 & 26,065 & 26,065 \\           
\hline\hline\addlinespace
Panel B: IHS(\# Hispanic Students Taking AP) & & & \\ 
\hline\addlinespace
 TV Dummy & 0.072$^{***}$ & 0.051$^{***}$ & 0.047$^{***}$ \\ 
  & (0.016) & (0.015) & (0.015) \\ 
 TV Dummy $\times$ Distance to Boundary & 0.002$^{***}$ & 0.002$^{***}$ & 0.003$^{***}$ \\ 
  & (0.0003) & (0.0003) & (0.0003) \\ 
 Distance to Boundary (meters) & $-$0.003$^{***}$ & $-$0.004$^{***}$ & $-$0.004$^{***}$ \\ 
  & (0.001) & (0.001) & (0.001) \\ 
 \# Hispanic Students & 0.002$^{***}$ & 0.001$^{***}$ & 0.001$^{***}$ \\ 
  & (0.00004) & (0.0001) & (0.0001) \\ 
Observations & 6,089 & 6,089 & 6,089 \\     
\hline\hline\addlinespace
Panel C: IHS(\# Hispanic Students Passing AP) &&& \\ 
\hline\addlinespace
 TV Dummy & 0.034$^{**}$ & 0.042$^{***}$ & 0.039$^{***}$ \\ 
  & (0.014) & (0.013) & (0.013) \\ 
 TV Dummy $\times$ Distance to Boundary & 0.0003 & 0.0003 & 0.0003 \\ 
  & (0.0003) & (0.0002) & (0.0002) \\ 
 Distance to Boundary (meters) & 0.002$^{**}$ & 0.002$^{*}$ & 0.001 \\ 
  & (0.001) & (0.001) & (0.001) \\ 
 \# Hispanic Students & 0.001$^{***}$ & 0.001$^{***}$ & 0.001$^{***}$ \\ 
  & (0.00003) & (0.00004) & (0.00004) \\ 
Observations & 2,205 & 2,205 & 2,205 \\                 
\hline\hline\addlinespace
                                County Controls & Yes & Yes  & Yes\\
                                School Size Controls & No & Yes & Yes\\
                                School Type Controls & No & No & Yes \\
				\addlinespace\hline\hline
			\end{tabular}
			\begin{tablenotes}[flushleft]
				\item \textit{Notes:} The table presents coefficient estimates from regressions at the school level, only keeping schools within 100 KM of a contour boundary. The dependent variables are inverse hyperbolic sine transformed counts of Hispanic students in gifted programs in Panel A, Hispanic students enrolled in AP courses in Panel B, and Hispanic students passing AP courses in Panel C. The key dependent variable of interest is the TV Dummy, which tracks whether the school is within a coverage contour boundary for a Spanish language television station. This is interacted with the distance to the boundary. County controls include log income, log population, and percentage county Hispanic for the county which the school is located in. School size controls account for the number of teachers and total number of students at the school, while school type controls include dummies for whether the school contains a primary, middle, and high school division. All regressions also control for the number of Hispanic students enrolled at the school. Standard errors are given in parentheses. *, **, and *** denote statistical significance at the 10\%, 5\%, and 1\% levels, respectively.
			\end{tablenotes}
		\end{threeparttable}
	}
\end{table}
%\begin{table}[!h]
	\centering
	\captionsetup{skip=1.5pt}
	\caption{Influence of Spanish Language Television on Hispanic Disciplinary Outcomes} \label{edu_bot}
	\scalebox{.8}{
		\begin{threeparttable}
			\begin{tabular}{lcccccccccc}
				\hline\hline\addlinespace
				Panel A: IHS(\# Hispanic Out of School Suspensions) &  (1) & (2) & (3) \\
                                \hline\addlinespace
 TV Dummy & $-$0.011$^{**}$ & $-$0.018$^{***}$ & $-$0.016$^{***}$ \\ 
  & (0.005) & (0.005) & (0.005) \\ 
 TV Dummy $\times$ Distance to Boundary & 0.0004$^{***}$ & 0.001$^{***}$ & 0.001$^{***}$ \\ 
  & (0.0001) & (0.0001) & (0.0001) \\ 
 Distance to Boundary (meters) & $-$0.002$^{***}$ & $-$0.002$^{***}$ & $-$0.002$^{***}$ \\ 
  & (0.0002) & (0.0002) & (0.0002) \\ 
 \# Hispanic Students & 0.003$^{***}$ & 0.002$^{***}$ & 0.002$^{***}$ \\ 
  & (0.00002) & (0.00003) & (0.00003) \\ 
Observations & 40,864 & 40,864 & 40,864 \\        
\hline\hline\addlinespace
Panel B: IHS(\# Hispanic Students Chronically Absent) & & & \\ 
\hline\addlinespace
 TV Dummy & $-$0.067$^{***}$ & $-$0.073$^{***}$ & $-$0.074$^{***}$ \\ 
  & (0.006) & (0.006) & (0.006) \\ 
 TV Dummy $\times$ Distance to Boundary & 0.001$^{***}$ & 0.001$^{***}$ & 0.001$^{***}$ \\ 
  & (0.0001) & (0.0001) & (0.0001) \\ 
 Distance to Boundary (meters) & $-$0.006$^{***}$ & $-$0.006$^{***}$ & $-$0.006$^{***}$ \\ 
  & (0.0003) & (0.0003) & (0.0003) \\ 
 \# Hispanic Students & 0.004$^{***}$ & 0.003$^{***}$ & 0.003$^{***}$ \\ 
  & (0.00003) & (0.00004) & (0.00004) \\ 
  Observations & 40,869 & 40,869 & 40,869 \\ 
\hline\hline\addlinespace
                                County Controls & Yes & Yes  & Yes\\
                                School Size Controls & No & Yes & Yes\\
                                School Type Controls & No & No & Yes \\
				\addlinespace\hline\hline
			\end{tabular}
			\begin{tablenotes}[flushleft]
				\item \textit{Notes:} The table presents coefficient estimates from regressions at the school level, only keeping schools within 100 KM of a contour boundary. The dependent variables are inverse hyperbolic sine transformed counts of Hispanic students who have received an out of school suspension in the prior year in Panel A, and Hispanic students chronically absent (over 15 days a year) in Panel B. The key independent variable of interest is the TV Dummy, which tracks whether the school is within a coverage contour boundary for a Spanish language television station. This is interacted with the distance to the boundary. County controls include log income, log population, and percentage county Hispanic for the county which the school is located in. School size controls account for the number of teachers and total number of students at the school, while school type controls include dummies for whether the school contains a primary, middle, and high school division. All regressions also control for the number of Hispanic students enrolled at the school. Standard errors are given in parentheses. *, **, and *** denote statistical significance at the 10\%, 5\%, and 1\% levels, respectively.
			\end{tablenotes}
		\end{threeparttable}
	}
\end{table}
%
\begin{table}[!htbp] \centering 
  \caption{Effect of Spanish language TV on Hispanic vs. Asian identity outcomes} 
  \label{t:edu_mech} 
	\scalebox{.8}{
		\begin{threeparttable}
			\begin{tabular}{lcccccccccc}
				\hline\hline\addlinespace
%				& \multicolumn{4}{c}{\textit{Minutes of TV watched}}  \\  				\cmidrule(lr){2-5} 
				&  (1) & (2) & (3)  \\
				\addlinespace\hline\addlinespace
				\multicolumn{3}{l}{Panel A: IHS(limited English proficiency)} \\ %edu_dda_satactOLSIHS_spec3
                              	\hline\addlinespace
				TV dummy $\times$ Hispanic & 0.3042$^{***}$ & 0.3042$^{***}$ & 0.3042$^{***}$\\
				  &(0.0379) & (0.0379) & (0.0379)\\
				\addlinespace\hline
				N & 83,004 & 83,004 & 83,004 \\ 
				\hline\hline\addlinespace
				\multicolumn{3}{l}{Panel B: IHS(bullied based on ethnicity)} \\ %edu_dda_calcOLSIHS_spec3
                              	\hline\addlinespace
				 TV dummy $\times$ Hispanic & 0.0015$^{*}$ & 0.0015$^{*}$ & 0.0015$^{*}$\\
				  &(0.0009) & (0.0009) & (0.0009)\\		 
				   \addlinespace\hline
				N & 52,068 & 52,068 & 52,068 \\ 
				\hline\hline\addlinespace
				School district FE & Yes & Yes  & Yes\\
				\# Hispanic, Asian students & Yes & Yes  & Yes\\
                                	School size controls & No & Yes & Yes\\
                                	School type controls & No & No & Yes \\
					\addlinespace\hline\hline
			\end{tabular}
			\begin{tablenotes}[flushleft]
				\item \textit{Notes:} The table presents coefficient estimates from regressions at the school-ethnicity level, only keeping schools within 100 KM of a Spanish language TV contour boundary. The dependent variable are inverse hyperbolic sine transformed counts of students classified as having limited English proficiency in Panel A and the number of students bullied on the basis of their ethnicity or race in Panel B. TV dummy is an indicator variable for a school with access to Spanish language television, which is interacted with an indicator for whether the demographic is Hispanic (the omitted group are Asians). Columns 1-3 control for the number of Hispanic and Asian students enrolled. Columns 2-3 control for the number of teachers and total number of students at the school. Column 3 controls for indicators denoting whether the school contains a primary, middle, and high school division. School district fixed effects are always included. Standard errors are clustered at the school district level. *, **, and *** denote statistical significance at the 10\%, 5\%, and 1\% levels, respectively.
			\end{tablenotes}
		\end{threeparttable}
	}
\end{table}
\begin{table}[!h]
	\centering
	\captionsetup{skip=1.5pt}
	\caption{Robustness of Influence of Spanish Language Television on Hispanic Students Passing the AP} \label{edu_ap_robust}
	\scalebox{.8}{
		\begin{threeparttable}
			\begin{tabular}{lcccccccccc}
				\hline\hline\addlinespace
				 & \multicolumn{6}{c}{\textit{IHS(\# Hispanic Students Passing AP)}}\\
				&  (1) & (2) & (3) & (4) & (5) & (6) \\
                                \hline\addlinespace
 TV Dummy & 0.039$^{***}$ & 0.049$^{***}$ & 0.044$^{***}$ & 0.044$^{***}$ & 0.036$^{***}$ & 0.032$^{*}$ \\ 
  & (0.013) & (0.017) & (0.016) & (0.017) & (0.013) & (0.018) \\ 
 TV Dummy $\times$ Distance to Boundary & 0.0003 & 0.0001 & 0.001 & 0.001$^{*}$ & 0.0001 & 0.001 \\ 
  & (0.0002) & (0.001) & (0.001) & (0.0004) & (0.0004) & (0.001) \\ 
 Distance to Boundary (meters) & 0.001 & 0.012$^{***}$ & 0.006$^{***}$ & 0.006$^{***}$ & 0.003$^{**}$ & 0.001 \\ 
  & (0.001) & (0.003) & (0.002) & (0.002) & (0.002) & (0.004) \\ 
 \# Hispanic Students & 0.001$^{***}$ & 0.001$^{***}$ & 0.001$^{***}$ & 0.001$^{***}$ & 0.001$^{***}$ & 0.001$^{***}$ \\ 
  & (0.00004) & (0.00004) & (0.00005) & (0.0002) & (0.00004) & (0.0001) \\ 
 Total APs Passed &  &  &  &  & 0.003$^{***}$ &  \\ 
  &  &  &  &  & (0.0001) &  \\ 
Observations & 2,205 & 2,205 & 1,525 & 1,525 & 1,525 & 1,095 \\ 
\hline\hline\addlinespace
                                County/School Controls & Yes & Yes  & Yes & Yes & Yes & Yes\\
                                Distance (KM) & 100 & 100 & 50 & 50 & 50 & 33 $\frac{1}{3}$ \\
                                Distance$^{2}$ Interaction & No & Yes & No & No & No & No \\
                                County F.E. & No & No & No & Yes & No & No & No \\
				\addlinespace\hline\hline
			\end{tabular}
			\begin{tablenotes}[flushleft]
				\item \textit{Notes:} The table presents coefficient estimates from regressions at the school level. The dependent variable is the inverse hyperbolic sine transformed counts of Hispanic students who have passed an AP exam. The key dependent variable of interest is the TV Dummy, which tracks whether the school is within a coverage contour boundary for a Spanish language television station. This is interacted with the distance to the boundary. County and school controls include log income, log population, percentage county Hispanic for the county which the school is located in, and the number of teachers, total number of students at the school, and dummies for whether the school contains a primary, middle, and high school division. Various distance cut-offs to the boundary are presented, as well as the TV dummy interacted with the square of the distance. All regressions also control for the number of Hispanic students enrolled at the school. Standard errors are given in parentheses. *, **, and *** denote statistical significance at the 10\%, 5\%, and 1\% levels, respectively.
			\end{tablenotes}
		\end{threeparttable}
	}
\end{table}
\begin{table}[!h]
	\centering
	\captionsetup{skip=1.5pt}
	\caption{Spatial Robustness of Influence of Spanish Language Television on Hispanic Victims of Ethnicity-Based Harassment} \label{edu_spatial}
	\scalebox{.8}{
		\begin{threeparttable}
			\begin{tabular}{lcccccccccc}
				\hline\hline\addlinespace
				 & \multicolumn{3}{c}{\textit{IHS(\# Hispanic Victims of Harassment)}}\\
				&  (1) & (2) & (3) \\
                                \hline\addlinespace
 TV Dummy & 0.003$^{**}$ & 0.002$^{***}$ & 0.003$^{*}$ \\ 
  & (0.001) & (0.001) & (0.002) \\ 
 TV Dummy $\times$ Distance to Boundary & $-$0.0001$^{**}$ & $-$0.0001$^{***}$ & $-$0.0001$^{**}$ \\ 
  & (0.00002) & (0.00001) & (0.00003) \\ 
\hline\hline\addlinespace
Observations & 40,811 & 40,811 & 40,811 \\ 
Log Likelihood &  & $-$4,304.916 & $-$4,299.820 \\ 
$\sigma^{2}$ &  & 0.072 & 0.072 \\ 
Akaike Inf. Crit. &  & 8,629.833 & 8,619.640 \\ 
Wald Test (df = 1) &  & 686.149$^{***}$ & 686.981$^{***}$ \\ 
LR Test (df = 1) &  & 657.312$^{***}$ & 667.505$^{***}$ \\ 
\hline \addlinespace
                                County/School Controls & Yes & Yes  & Yes \\
                                Model & OLS & SAR Lag & SAR Error \\
				\addlinespace\hline\hline
			\end{tabular}
			\begin{tablenotes}[flushleft]
				\item \textit{Notes:} The table presents coefficient estimates from regressions at the school level, only keeping schools within 100 KM of a contour boundary. The dependent variable is the inverse hyperbolic sine transformed counts of Hispanic students who are bullied or harassed on the basis of their ethnicity. The key dependent variable of interest is the TV Dummy, which tracks whether the school is within a coverage contour boundary for a Spanish language television station. This is interacted with the distance to the boundary. County and school controls include log income, log population, percentage county Hispanic for the county which the school is located in, and the number of teachers, total number of students at the school, and dummies for whether the school contains a primary, middle, and high school division. The SAR Lag model is a spatial autoregressive lag model and the SAR Error model is a spatial autoregressive error model, both with weight matrices based on 4 nearest neighbours. Standard errors are given in parentheses. *, **, and *** denote statistical significance at the 10\%, 5\%, and 1\% levels, respectively.
			\end{tablenotes}
		\end{threeparttable}
	}
\end{table}

%\begin{table}[!h]
	\centering
	\captionsetup{skip=1.5pt}
	\caption{Influence of Spanish Language Television on Campaign Contributions} \label{don_base}
	\scalebox{.8}{
		\begin{threeparttable}
			\begin{tabular}{lcccccccccc}
				\hline\hline\addlinespace
				& \multicolumn{4}{c}{\# Hispanic Campaign Contributions} \\
				\cline{2-5} 
				Panel A: Contributions to Trump &  (1) & (2) & (3) & (4) \\
                                \hline\addlinespace
 TV Dummy & 0.019$^{***}$ & 0.010$^{***}$ & 0.007$^{***}$ & 0.005$^{***}$ \\ 
  & (0.001) & (0.001) & (0.001) & (0.001) \\ 
 TV Dummy $\times$ Distance to Boundary  & 0.002$^{***}$ & 0.001$^{***}$ & 0.001$^{***}$ & 0.001$^{***}$ \\ 
  & (0.0001) & (0.0001) & (0.0001) & (0.0001) \\ 
 Distance to Boundary (KM) & 0.0001 & 0.0003$^{***}$ & 0.0003$^{***}$ & 0.0004$^{***}$ \\ 
  & (0.0001) & (0.0001) & (0.0001) & (0.0001) \\ 
 Log(Population) &  & 0.081$^{***}$ & 0.084$^{***}$ & 0.058$^{***}$ \\ 
  &  & (0.001) & (0.001) & (0.001) \\ 
 County \% Hispanic &  &  & 0.084$^{***}$ & 0.265$^{***}$ \\ 
  &  &  & (0.007) & (0.008) \\ 
 Log(Income) &  &  &  & 0.00003$^{***}$ \\ 
  &  &  &  & (0.00000) \\ 
Observations & 619,011 & 619,011 & 619,011 & 619,011 \\ 
\hline\addlinespace
Panel B: Contributions to Clinton & & & \\ 
\hline\addlinespace
 TV Dummy & $-$0.008$^{**}$ & $-$0.014$^{***}$ & $-$0.019$^{***}$ & $-$0.020$^{***}$ \\ 
  & (0.004) & (0.004) & (0.004) & (0.004) \\ 
 TV Dummy $\times$ Distance to Boundary  & 0.003$^{***}$ & 0.002$^{***}$ & 0.002$^{***}$ & 0.002$^{***}$ \\ 
  & (0.0001) & (0.0001) & (0.0001) & (0.0001) \\ 
 Distance to Boundary (KM) & 0.0002 & 0.0004$^{**}$ & 0.0004$^{***}$ & 0.0004$^{***}$ \\ 
  & (0.0001) & (0.0001) & (0.0001) & (0.0001) \\ 
 Log(Population) &  & 0.053$^{***}$ & 0.056$^{***}$ & 0.038$^{***}$ \\ 
  &  & (0.003) & (0.003) & (0.003) \\ 
 County \% Hispanic &  &  & 0.106$^{***}$ & 0.229$^{***}$ \\ 
  &  &  & (0.019) & (0.022) \\ 
 Log(Income) &  &  &  & 0.00002$^{***}$ \\ 
  &  &  &  & (0.00000) \\ 
Observations & 619,011 & 619,011 & 619,011 & 619,011 \\ 
				\addlinespace\hline\hline
			\end{tabular}
			\begin{tablenotes}[flushleft]
				\item \textit{Notes:} The table presents coefficient estimates from regressions that divide the US up into grid points of size 4 KM$^2$. The dependent variables are the summed counts of Hispanic contributions to political campaigns in the 2016 presidential election. The key dependent variable of interest is the TV Dummy, which tracks whether the destination county is inside or outside the TV contour. This is interacted with the distance to the boundary. County controls include log income, log population, and percentage county Hispanic. Standard errors are given in parentheses. *, **, and *** denote statistical significance at the 10\%, 5\%, and 1\% levels, respectively.
			\end{tablenotes}
		\end{threeparttable}
	}
\end{table}


%\begin{table}[!h]
%	\centering
%	\captionsetup{skip=1.5pt}
%	\caption{Classification of Hispanic Firm Names} \label{firm_class}
%	\scalebox{.8}{
%		\begin{threeparttable}
%			\begin{tabular}{llcccccccc}
%				\hline\hline\addlinespace
%    \textit{Section A: Latin American Countries} & mexic, bolivia, chile, argentin, venezuela, beliz, costa rica, salvador, guatemala \\
%&  hondur, nicaragua, panama, brazil, colombia, ecuador, guyana, paragua, peru \\
%& surinam, urugu, cuba, dominican, haiti, puerto, latin \\
%				\addlinespace\hline
%\textit{Section B: Common Spanish Words} & la, de, como, su, que, el, para, en, por, los, casa, caliente \\ %%% MOST COMMON 
%\hline
%\textit{Section C: Common Hispanic Foods} & taqueria, taco, empanada, huevo, pollo, burrito, arepa, pupusa, tamale, tortilla \\
%& salsa, asado, lechon, mojo, ropa, vieja, chorizo \\		 		
%				\addlinespace\hline\hline
%			\end{tabular}
%			\begin{tablenotes}[flushleft]
%				\item \textit{Notes:} Firms are classified as having a Hispanic name if any keyword in the table above is matched within the word, subject to the following requirements: (1) Case does not matter, (2) For Sections A and C, an exact match of the string at any location in the firm name, (3) For Section B, the string must be a distinct word, (4) The inclusion of Section C is optional.
%			\end{tablenotes}
%		\end{threeparttable}
%	}
%\end{table}


%\
%
%\includegraphics[scale=1]{fig_1_2017.eps}
%
%\settowidth{\tableboxwidth}{\usebox{\tablebox}} \parbox{.92\textwidth}
%{\emph{Notes}: the two bars on the left display donation rates to the anti-immigration organization for individuals in the private and public conditions in the control group before the election (full sample, respectively N=112 and N=111), the two central bars display those in the information group before the election (full sample, respectively N=102 and N=103), and the last two bars display those in the control group after the election (for individuals already surveyed before the election, respectively N=82 and N=84). Error bars reflect 95\% confidence intervals. Top horizontal bars show \emph{p}-values for \emph{t} tests of equality of means between different experimental conditions.}
%\end{center}
%\end{figure}

\pagebreak




\end{document}


%Resources:
%https://rpubs.com/chrisbrunsdon/114718
%https://rpubs.com/corey_sparks/109650
%http://www.econ.uiuc.edu/~lab/workshop/Spatial_in_R.html
%https://rspatial.org/raster/analysis/7-spregression.html
%Conley Errors





 