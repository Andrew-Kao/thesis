

\documentclass[11pt]{article}
%%%%%%%%%%%%%%%%%%%%%%%%%%%%%%%%%%%%%%%%%%%%%%%%%%%%%%%%%%%%%%%%%%%%%%%%%%%%%%%%%%%%%%%%%%%%%%%%%%%%%%%%%%%%%%%%%%%%%%%%%%%%%%%%%%%%%%%%%%%%%%%%%%%%%%%%%%%%%%%%%%%%%%%%%%%%%%%%%%%%%%%%%%%%%%%%%%%%%%%%%%%%%%%%%%%%%%%%%%%%%%%%%%%%%%%%%%%%%%%%%%%%%%%%%%%%
\usepackage{amsmath,amsthm,amssymb, pdfpages,mathtools}
\usepackage{color}
\usepackage{array}
\usepackage{gastex}
\usepackage{subfigure}
\usepackage[normalem]{ulem}
\usepackage{xcolor,psfrag,graphicx}
\usepackage{setspace}
\usepackage{natbib}
\usepackage{lscape}
\usepackage{enumerate}
\usepackage{appendix}
\usepackage[hidelinks, hypertexnames=false]{hyperref}
\usepackage{lscape}

\setcounter{MaxMatrixCols}{10}

\setlength{\evensidemargin}{0.0in}
 \setlength{\oddsidemargin}{0.0in}
 \setlength{\textwidth}{6.5in}
 \topmargin -0.25in
 \textheight 8.5in
 \hfuzz=50pt
 \pagestyle{plain}
\newcommand{\eqthreshn}{{t^*_N}}
\newcommand{\pnd}{1-p+pF(\eqthreshn)}
\newcommand{\ppnd}{\big(1-p+pF(\eqthreshn)\big)}
\newcommand{\eqmfreq}{{\omega^*}}
\newcommand{\eqmfreqn}{{\omega^*_N}}
\newcommand{\eqmfreqnp}{{\omega^*_{N+1}}}
\newcommand{\eqthresh}{{t^*}}
\newcommand{\eqthreshX}{{t^{**}}}
\newcommand{\nbar}{{\overline{N}}}
\newcommand{\wlim}{\omega_\infty}
\newcommand{\wdye}{\hat{\omega}}
\newcommand{\tdye}{\hat{t}}
\newcommand{\limn}{\lim_{N\to\infty}}
\newcommand{\fsn}{\omega_N^*}
\newcommand{\eqprize}{\phi^*}
\newcommand{\moprize}{\phi^M}
\newcommand{\dif}{\;\mathrm{d}}
\newcommand{\diffp}[2]{\frac{\partial #1}{\partial #2}}
\newcommand{\diff}[2]{\frac{\dif #1}{\dif #2}}
\renewcommand{\Re}{\mathbb{R}}                             
\def\endproof{{\quad}$\blacksquare$}
\newcommand{\indicator}[1]{\mathbbm{1}_{\left[ {#1} \right]}}
\newtheorem{theorem}{Theorem}
\newtheorem{proposition}{Proposition}
\newtheorem{prop}{Proposition}
\newtheorem{example}{Example}
\newtheorem{assumption}{Assumption}
\newtheorem{corollary}[theorem]{Corollary}
\newtheorem{acknowledgement}[theorem]{Acknowledgement}
\newtheorem{definition}{Definition}
\newtheorem{lemma}{Lemma}
\newtheorem{remark}{Remark}
\newtheorem{condition}[theorem]{Condition}
 \setlength{\evensidemargin}{0.0in}
 \setlength{\oddsidemargin}{0.0in}
 \setlength{\textwidth}{6.5in}
 \topmargin -0.25in
 \textheight 8.5in
 \hfuzz=50pt
 \pagestyle{plain}
\newcommand{\Change}[1]{{\color{red}#1}}
\renewcommand{\theenumi}{\roman{enumi}}            
\renewcommand{\labelenumi}{(\theenumi)}

%%%%%%%%%%%%%%%
% Page Format %
%%%%%%%%%%%%%%%
 \setlength{\evensidemargin}{0.0in}
 \setlength{\oddsidemargin}{0.0in}
 \setlength{\textwidth}{6.5in}
 \topmargin -0.25in
 \textheight 8.5in
 \hfuzz=50pt
 \pagestyle{plain}





\begin{document}
\title{\textbf{TV Identities: \\ TITLE}%Unravel}
\thanks{We would like to thank XYZ.}\\
}



\author{Andrew Kao\thanks{University of Chicago, andrewkao@uchicago.edu.} }

%\begin{center}
\date{January 2020}
{\vspace{-5ex}}
%\end{center}


\maketitle

\begin{abstract}
%\noindent When deciding whether to publicly express their views and attitudes, individuals may take into account their perception of the social acceptability of these views, which in turn depends on (and affects) their perception of the popularity of the view. Even if a substantial fraction of society \emph{privately} holds a specific view, that view might end up stigmatized and publicly withheld in equilibrium if individuals have initial beliefs that the view is stigmatized. 

%%%% FOR SUBMISSION FORM:
%\noindent Social norms, usually persistent, can unravel quickly when new public information arrives, such as surprising election outcomes. In our model of strategic communication, senders state their opinion but they can lie to pander to the popular view; receivers thus make less inference about such senders. We test the model's predictions with two experiments. On the sender's side, we show that Trump's rise in popularity and eventual victory increased individuals' willingness to publicly express xenophobic views. On the receiver's side, we show that individuals are judged less if they expressed a xenophobic view in an environment where the view is popular.


\noindent Here's an abstract
\noindent
\\
%\textbf{JEL Codes:} D1, I31, Z13.\\
%\textbf{Keywords:} 
\end{abstract}




\newsavebox{\tablebox} \newlength{\tableboxwidth}

\setlength{\baselineskip}{22pt}

\renewcommand{\thefootnote}{\fnsymbol{footnote}}


\thispagestyle{empty}

\newpage 
\renewcommand{\thefootnote}{\arabic{footnote}}

\pagebreak 
\setcounter{page}{0}


\onehalfspacing

%\tableofcontents

\newpage

\setcounter{page}{1}
\section{Introduction}

Social norms are an important element of any society: some behaviors and opinions are socially desirable, while others are stigmatized. There is growing evidence that individuals care to a large extent about how they are perceived by others and that such concerns might affect important decisions in a variety of settings

% viewed by half of all Spanish-dominant Latinos (http://www.horowitzresearch.com/press/spanish-language-tv-content-remains-integral-to-u-s-hispanics-tv-diet-new-horowitz-survey-shows/)
% better: Nielsen, 78% Spanish-dominant watch Spanish TV, 50% in multi-language homes, over 85% broadcast -- in 2010, top 10 broadcast shows in Hispanic demographic were all Spanish language
% market size: millions in LA, New York etc. https://www.statista.com/statistics/189824/largest-hispanic-television-markets-in-the-united-states-2011/

% broadcast/satellite TV vs cable: https://www.quora.com/What-is-the-difference-between-cable-and-broadcast

% in recent years, highest viewed: Peque�os Gigantes (talent show, kids), El Se�or de los Cielos (telenovella, cartel leader),   https://www.statista.com/statistics/497739/spanish-tv-programs-usa/

% \citep{andreoni_bernheim_2009, dellavigna_list_malmendier_2012_testing_altruism, andreoni_rao_trachtman_2017} to schooling choices (\citealp{bursztyn_jensen_2015_peer_education}) to political behavior (\citealp{gerber_green_larimer_2008, dellavigna_list_malmendier_rao_2017_voting, enikolopov_makarin_petrova_plishchuk_2017_social_image, ricardo_cruces_2017}). 

%\footnote{% We thus focus on the consequences of Trump's election rather than its causes or
%determinants. With respect to the latter, \citet{enke_2017_trump}
%demonstrates the link between tribalistic (as opposed to universal) moral
%values and Trump vote at the county level, while %
%\citet{allcott_gentzkow_2017_fake_news} discuss the possible role of fake
%news. Relatedly, \citet{xiong_2017_personality} studies the effect of the
%celebrity status of Ronald Reagan on his electoral support, and suggests
%that a similar effect may have helped Trump. At the same time, our focus is
%on causes and not consequences of changes in social norms (see %
%\citealp{ali_benabou_2016}, on the latter).}

%% Do everything with d^2
%% spatial errors: http://www.trfetzer.com/using-r-to-estimate-spatial-hac-errors-per-conley/

\section{Motivating Framework} \label{sectheory}

To organize thoughts and motivate our experimental designs, we present a simple model of communication.


\subsection{Model}

\subsection{Analysis}

\subsection{Discussion}



\section{Experiment 1: Expressing Xenophobia} \label{secexp1}

In this section we present the results of two related experiments showing that Donald Trump's rise in popularity and eventual victory in the 2016 U.S. Presidential election causally increased individuals' perception of the social acceptability of holding strong anti-immigration views and their willingness to publicly express them.




\subsection{Experiment 1A: U.S. Presidential Elections} 

We implemented the first experiment respectively in the two weeks before and in the week after the 2016 presidential election. The timing of the experiment allowed us to exploit the uniqueness of the situation and study the process of information aggregation as it was unfolding. We conducted both waves with workers from the online platform MTurk. A number of recent papers in economics have used the same platform to conduct surveys or experiments (e.g., \citealp{kuziemco_et_al_2015_elastic}). The platform draws workers from very diverse backgrounds, though it is not representative of the U.S. population as a whole.

\subsubsection{Experimental Design}


\paragraph{Wave 1: Intervention Before the Election.}

\subsubsection{Main Results} \label{results}

%Appendix Table \ref{balance1} provides evidence that individual characteristics are balanced across all four pre-election experimental conditions, confirming that the randomization was successful. The first four bars of Figure \ref{mainexperiment1fig} display our main findings from the pre-election experiment. In the control condition before the election, we observe a large and statistically significant wedge between donation rates in private and in public: a drop from 54\% in private to 34\% in public (the p-value of a \emph{t} test of equality is 0.002). Among individuals in the information condition, we observe no difference in private and public donation rates, which are 47\% and 46\%, respectively (\emph{p}-value=0.839). Moreover, we find no significant difference in private donation rates between the information and control conditions (\emph{p}-value=0.280), suggesting that the information is not increasing privately-held xenophobia. The increase in public donation rates between the two conditions is statistically significant (\emph{p}-value=0.089), as is the difference in differences between donation rates in private across conditions and donation rates in public across conditions (\emph{p}-value=0.050). These results indicate that the information provided causally increased the social acceptability of the action to the point of eliminating the original social stigma associated with it.\footnote{Apart from social stigma, another possible reason %An additional explanation to social stigma 
%for the lower donation rates in the public condition with respect to the private condition is that participants might want to avoid talking with the surveyor because of the extra effort and time this requires (independently of the topic of the conversation), and they might expect the likelihood of having to talk to be higher in case they decide to make the donation. However, this mechanism should operate identically both in the control and in the treatment conditions, thus not affecting our identification of the reduction in social stigma.} The first two columns of Table \ref{mainexperiment1tab} display the difference in differences results in regression format and show that our results are unchanged when individual covariates are included. The table also displays \emph{p}-values from permutation tests, showing that our findings are robust to that inference method.

\subsection{Evidence of Mechanism}



\section{Conclusion} \label{secconcl}


\clearpage
\pagebreak

\begin{singlespace}
\bibliographystyle{aea}
%\bibliography{xenophobia_bib2}
\end{singlespace}

\pagebreak
\clearpage

\section*{Figures and Tables}

%%%%%%%%%%%%%%%%%%%%%%%%%%%%%%%%%
% Figure 1: MAIN EXPERIMENT
%%%%%%%%%%%%%%%%%%%%%%%%%%%%%%%%%

%\begin{figure}[!h]
%\begin{center}
%\caption{{\bf Proposition 3: Receiver's Priors and Posteriors}} \label{theoryfig1}
%
%\
%
%\includegraphics[scale=.35]{Figure1Updated.pdf}
%
%\end{center}
%\end{figure}

%\begin{figure}[!h]
%\begin{center}
%\caption{{\bf Experiment 1A: Donation Rates Before and After the Election}} \label{mainexperiment1fig}
%
%\
%
%\includegraphics[scale=1]{fig_1_2017.eps}
%
%\settowidth{\tableboxwidth}{\usebox{\tablebox}} \parbox{.92\textwidth}
%{\emph{Notes}: the two bars on the left display donation rates to the anti-immigration organization for individuals in the private and public conditions in the control group before the election (full sample, respectively N=112 and N=111), the two central bars display those in the information group before the election (full sample, respectively N=102 and N=103), and the last two bars display those in the control group after the election (for individuals already surveyed before the election, respectively N=82 and N=84). Error bars reflect 95\% confidence intervals. Top horizontal bars show \emph{p}-values for \emph{t} tests of equality of means between different experimental conditions.}
%\end{center}
%\end{figure}

\pagebreak



%\begin{table}[!h]
%  \centering
%\caption{{\bf Experiment 1A: Difference in Differences Regressions}\label{mainexperiment1tab}} 
%      \begin{lrbox}{\tablebox}
%\begin{tabular}{l*{4}{c}}
%\\
%\hline
%            Dependent&\multicolumn{4}{l}{Dummy: individual authorizes donation to}\\
%                        Variable&\multicolumn{4}{l}{anti-immigrant  organization}\\
%\hline
%            &\multicolumn{1}{c}{(1)}&\multicolumn{1}{c}{(2)}&\multicolumn{1}{c}{(3)}&\multicolumn{1}{c}{(4)}\\
%\hline
%Public & -0.202*** & -0.200*** & -0.202*** & -0.199***\\
%& [0.065] & [0.066] & [0.065] & [0.065]\\
%& (0.004) & (0.005) & (0.004) & (0.007)\\
%Information & -0.074 & -0.077 & -0.074 & -0.076\\
%& [0.069] & [0.068] & [0.069] & [0.068]\\
%& (0.277) & (0.266) & (0.277) & (0.281)\\
%Public*Information & 0.188* & 0.178* & 0.188* & 0.178*\\
%& [0.096] & [0.096] & [0.096] & [0.096]\\
%& (0.045) & (0.062) & (0.045) & (0.062)\\
%After Election & & & -0.057 & -0.062\\
%& & & [0.073] & [0.072]\\
%& & & (0.380) & (0.304)\\
%Public*After Election & & & 0.191* & 0.186*\\
%& & & [0.102] & [0.101]\\
%& & & (0.071) & (0.080)\\
%Mean Donation Rate\\
%Control Private & \multicolumn{4}{c}{0.545}\\
%Before Election\\
%\hline
%Controls & No & Yes & No & Yes\\
%N & 428 & 428 & 594 & 594\\
%$R^{2}$ & 0.022 & 0.033 & 0.017 & 0.034\\
%\hline
%\end{tabular}
%   \end{lrbox}
%
%   \usebox{\tablebox}\\
%\settowidth{\tableboxwidth}{\usebox{\tablebox}} \parbox{.995\tableboxwidth}{\emph{Notes}: Columns (1) and (2) includes the full pre-election sample. Columns (3) and (4) add the post-election sample of individuals already surveyed before the election. Columns (1) presents OLS regression of a dummy variable for whether a individual donates to the anti-immigration organization on a dummy for the Public condition, a dummy for the Information condition, and a dummy for the Public Information condition. The control private condition before the election is the omitted group, for which we report the mean donation rate. Columns (3) replicates and adds a dummy for the after election condition, and a dummy for the Public after election condition. Columns (2) and (4) replicate and add individual covariates (gender, age, marital status, years of education, household income, and race). Robust standard errors in brackets. \emph{P}-values from permutation tests with 1,000 repetitions in parentheses. * significant at 10\%; ** significant at 5\%; *** significant at 1\% based on robust standard errors.}
%%\end{tiny}
%\end{table}
%

\end{document}

